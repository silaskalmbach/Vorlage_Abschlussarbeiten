\chapter*{Eidesstattliche Erklärung}
% Falls Kapitel nicht im Inhaltsverzeichnis erscheinen soll, folgende Zeile auskommentieren
\addcontentsline{toc}{chapter}{Eidesstattliche Erklärung}
\label{erklaerung}
Hiermit erkläre ich, dass ich die vorliegende Arbeit selbständig verfasst habe, dass ich keine anderen als die angegebenen Quellen benutzt und alle wörtlich oder sinngemäß aus anderen Werken übernommenen Aussagen als solche gekennzeichnet habe, dass die eingereichte Arbeit weder vollständig noch in wesentlichen Teilen Gegenstand eines anderen Prüfungsverfahrens gewesen ist, dass ich die Arbeit weder vollständig noch in Teilen bereits veröffentlicht habe und dass das elektronische Exemplar mit den anderen Exemplaren übereinstimmt. \\
\\[1.5cm]
Datum:	\hrulefill\enspace Unterschrift: \hrulefill
\\[3.5cm]

\vfill

Bitte zitieren Sie diese Arbeit unter Verwendung des folgenden RIS-Eintrages:

\texttt{TY  - THES \\
TI  - \TitelDerArbeit \\
AU  - \StudentNachname, \StudentVorname \\
CN  - \NummerDerArbeit/\StrRight{\EndeJahr}{2} \\
DA  - \Abgabedatum \\
PY  - \EndeJahr \\
M3  - \TypDerArbeit \\
CY  - Stuttgart \\
PB  - Institut für Leichtbau Entwerfen und Konstruieren \\
N1  - \Betreuer
}
