% ============= Einstellungen zur Arbeit =============
\documentclass[fontsize=11pt]{scrartcl}


% Variablen einbinden
%% ============= Einstellungen Abmessungen Einband ============= 

% Breite des Buchrückens in mm, dies anpasse, siehe auch ReadMe
\newcommand\spineWidth{10}

% Hier muss nichts verändert werden
% DIN A4 Abmessungen
\newcommand\standardPageWidth{210}
\newcommand\standardPageHeight{297}

% Abmessungen Einband
\newcommand\bookCoverWidth{\numexpr\spineWidth + 2*\standardPageWidth\relax}
\newcommand\bookCoverHeight{\standardPageHeight}

% ============= Angaben zur Arbeit =============

% Hier den Titel der Arbeit angeben 
\newcommand{\TitelDerArbeit}{Xxxxx Xxxxx Xxxxx Xxxxx Xxxxx Xxxxx Xxxxx Xxxxx Xxxxx} 

% Hier die Art der Abschlussarbeit eingeben 
\newcommand{\TypDerArbeit}{Masterarbeit} 

% Nummerrierung der Arbeit (mit Betreuer zu klären)
\newcommand{\NummerDerArbeit}{XX} 

% Angabe des Monates der Abgabe als Wort
\newcommand{\EndeMonat}{Januar} 

% Angabe des Jahres der Abgabe
\newcommand{\EndeJahr}{20XX} 

% Hier den Betreuer der Arbeit angeben
\newcommand{\Betreuer}{Xxxxx Xxxxx} 

% Hier den Prüfer der Arbeit angeben 
\newcommand{\Pruefer}{Xxxxx Xxxxx} 

%Hier den Studierendenname der Arbeit angeben 
\newcommand{\StudentVorname}{Max} 
\newcommand{\StudentNachname}{Mustermann} 

% Hier den Professor1 des Instituts angeben 
\newcommand{\ProfEins}{Prof. Dr.-Ing. M.Arch. Lucio Blandini} 

% Hier den Professor2 des Instituts angeben 
\newcommand{\ProfZwei}{Prof. Dr.-Ing. Balthasar Nov\'{a}k} 

% Hier das Institut angeben
\newcommand{\Institut}{Institut f{\"u}r Leichtbau Entwerfen und Konstruieren} 
% ============= Einstellungen Abmessungen Einband ============= 

% Breite des Buchrückens in mm, dies anpasse, siehe auch ReadMe
\newcommand\spineWidth{10}

% Hier muss nichts verändert werden
% DIN A4 Abmessungen
\newcommand\standardPageWidth{210}
\newcommand\standardPageHeight{297}

% Abmessungen Einband
\newcommand\bookCoverWidth{\numexpr\spineWidth + 2*\standardPageWidth\relax}
\newcommand\bookCoverHeight{\standardPageHeight}

% ============= Angaben zur Arbeit =============

% Hier den Titel der Arbeit angeben 
\newcommand{\TitelDerArbeit}{Xxxxx Xxxxx Xxxxx Xxxxx Xxxxx Xxxxx Xxxxx Xxxxx Xxxxx} 

% Hier die Art der Abschlussarbeit eingeben 
\newcommand{\TypDerArbeit}{Masterarbeit} 

% Nummerrierung der Arbeit (mit Betreuer zu klären)
\newcommand{\NummerDerArbeit}{XX} 

% Angabe des Monates der Abgabe als Wort
\newcommand{\EndeMonat}{Januar} 

% Angabe des Jahres der Abgabe
\newcommand{\EndeJahr}{20XX} 

% Hier den Betreuer der Arbeit angeben
\newcommand{\Betreuer}{Xxxxx Xxxxx} 

% Hier den Prüfer der Arbeit angeben 
\newcommand{\Pruefer}{Xxxxx Xxxxx} 

%Hier den Studierendenname der Arbeit angeben 
\newcommand{\StudentVorname}{Max} 
\newcommand{\StudentNachname}{Mustermann} 

% Hier den Professor1 des Instituts angeben 
\newcommand{\ProfEins}{Prof. Dr.-Ing. M.Arch. Lucio Blandini} 

% Hier den Professor2 des Instituts angeben 
\newcommand{\ProfZwei}{Prof. Dr.-Ing. Balthasar Nov\'{a}k} 

% Hier das Institut angeben
\newcommand{\Institut}{Institut f{\"u}r Leichtbau Entwerfen und Konstruieren} 


% ============= Einstellungen zum Kompilieren =============
\usepackage[%
paperheight=\bookCoverHeight mm,
paperwidth=\bookCoverWidth mm,
left= 0cm, right = 0cm, top=0cm, bottom=0cm,
showframe
]{geometry}

% Zeilenabstand zurücksetzen
\usepackage{setspace}
\setstretch{1.00}

% Update graphicspath to image folder
\usepackage{graphicx} %Einbinden von Bildern
%\graphicspath{{./01_images/}{./01_images/Titelseite/}} %Bilderordner
\graphicspath{{../../_E_ressources/images}} %Bilderordner
\usepackage{color}      %Verwendung von Farben
\usepackage{xcolor}		%Definition eigener Farben

% Uni Stuttgart Colors
\definecolor{uniSlightblue}{RGB}{0,190,255}
\definecolor{uniSdarkblue}{RGB}{0,65,145}
\definecolor{uniSdarkgrey}{RGB}{62, 68, 76}
\definecolor{uniSgreyblue}{RGB}{125,198,234}
\definecolor{uniSgrey}{RGB}{159,153,152}

% Uni Stuttgart monochrom greyscale
\definecolor{uniSblack}{RGB}{0,0,0}
\definecolor{uniSdarkgrey}{RGB}{62,68,76}
\definecolor{uniSgrey}{RGB}{159,153,152}
\definecolor{uniSlightgrey}{RGB}{185,186,187}

\usepackage{setspace}

\usepackage{ifluatex}
\ifluatex
\usepackage[utf8]{luainputenc} %Schriftkodierung LuaLaTeX
\else
\usepackage[utf8]{inputenc} %Schriftkodierung PDFLaTeX
\fi


% ============= Schriftarten =============

\ifluatex

\openin15=00_ressources/fonts/UniversforUniS55Rm-Regular.ttf   % TeX primitive level % \IfFileExists{filename}{true-branch}{false-branch} %LaTeX -> geht auch 15 ist eine Variable/Beschreibung
\ifeof15           %... file does not exist

\usepackage[no-math]{fontspec}
\setmainfont{Arial}

\addtokomafont{disposition}{\rmfamily} %Überschriften selbe Schriftart wie Text
\renewcommand\familydefault\sfdefault % Formeln in Sans-Serif
\usepackage[italic, symbolgreek]{mathastext} % Formeln in Sans-Serif
\renewcommand\familydefault\rmdefault % Formeln in Sans-Serif

\else\closein15    %... file exists

%% ============= Lua-LaTeX =============
\usepackage[no-math]{fontspec} 
\setmainfont{UniversforUniS}[
Extension      = .ttf,
Path           = 00_ressources/fonts/,
UprightFont    = *55Rm-Regular,
ItalicFont     = *45LtObl-Rg,
BoldFont       = *65Bd-Regular,
BoldItalicFont = *45LtObl-Rg,
]
\setsansfont{UniversforUniS}[
Extension      = .ttf,
Path           = 00_ressources/fonts/,
UprightFont    = *55Rm-Regular,
ItalicFont     = *45LtObl-Rg,
BoldFont       = *65Bd-Regular,
BoldItalicFont = *45LtObl-Rg,
]
%\setmonofont{}[]
\addtokomafont{disposition}{\rmfamily} %Überschriften selbe Schriftart wie Text
\renewcommand\familydefault\sfdefault % Formeln in Sans-Serif
\usepackage[italic, symbolgreek]{mathastext} % Formeln in Sans-Serif
\renewcommand\familydefault\rmdefault % Formeln in Sans-Serif
\fi

\else
%% ============= PDF-LaTeX =============
\usepackage[T1]{fontenc} %Richtige Trennung von Umlauten
\usepackage[scale=1]{tgheros}
%	\usepackage{fourier}
\renewcommand\familydefault\sfdefault % alles in Sans-Serif
\usepackage[italic, symbolgreek]{mathastext} % Formeln in Sans-Serif
\fi

% begin the document and suppress page numbers
\begin{document}
	
	\pagestyle{empty} 
	
	\begingroup%
	\makeatletter%
	% Conversion of pt to mm
	% #1: macro, which gets the result of the conversion without unit
	% #2: length expression
	\newcommand*{\converttomm}[2]{%
		\edef#1{%
			\strip@pt\dimexpr(#2)*2540/7227\relax % 72.27 pt = 1 in = 25.4 mm
		}%
	}
	\makeatother%
	
	% Folgende Maßeinheiten in mm
	\setlength{\unitlength}{1mm}
	
	\noindent\begin{picture}(\bookCoverWidth,\bookCoverHeight)(0,0)
		\converttomm{\fiveBaselineskip}{5.25\baselineskip}%
		
		% Prinzipielle Strategie: Mittels \put{x,y} [ab Eckpunkt unten links, Angaben in mm] die untere Linke Ecke einer Box erstellen. 
		% Mittels \makebox(Breite,Höhe) eine Box zwischen den Hilfslinien erstellen. Kann bspw. mit \framebox angezeigt werden.
		% In der Box dann Text und Bilder vertikal zentriert ausrichten
		% Die Positions ist abhängig von der Dicke des Buchrückens gesetzt (Siehe x-Koordinaten)
		
		% ILEK Logo Vorderseite
		\put(\numexpr\standardPageWidth + \spineWidth + 140\relax, 227){
			\makebox(0,45)[lc]{%
				\includegraphics[height=10mm]{ILEK-logo.jpg}
		}}
		
		% ILEK Logo Rückseite
		\put(28,227){
			\makebox(0,45)[lc]{%
				\includegraphics[height=10mm]{ILEK-logo}
		}}
		
		% Logo Uni Stuttgart und Beschriftung
		\put(\numexpr\standardPageWidth + \spineWidth + 20\relax,30){
			\makebox(0,25)[lc]{%
				\includegraphics[height=4.2\baselineskip]{unistuttgart_logo_de}
		}}
		
		\put(\numexpr\standardPageWidth + \spineWidth + 20\relax,30){% Position Text
			\put(\fiveBaselineskip,0){% 5x Texthöhe hinzufügen
				\makebox(0,25)[lc]{%
					\begin{tabular}{l}
						{\small \Institut} \\ 
						{\small \ProfEins} \\
						{\small \ProfZwei} \\
						{\small Universität Stuttgart}
					\end{tabular}
		}}}
		
		% Beschriftung des Buchrückens
		\put(\numexpr\standardPageWidth + \spineWidth / 2\relax, 42.5){% Position Text
			\rotatebox{90}{\makebox(0,0)[c]{
					\MakeUppercase{\EndeJahr~|~\NummerDerArbeit}
			}}
		}%
		
		\put(\numexpr\standardPageWidth + \spineWidth / 2\relax, 60.0){% Position Text
			\rotatebox{90}{\makebox(0,0)[l]{%
					\MakeUppercase{\StudentNachname\:-\:\TitelDerArbeit}
			}}
		}%
		
		\put(\numexpr\standardPageWidth + \spineWidth / 2\relax, 249.5){% Position Text
			\rotatebox{90}{\makebox(0,0)[c]{%
					\MakeUppercase{\TypDerArbeit}
			}}
		}%
		
		% Horizontale Linien hinzufügen
		\put(0,30){\color{uniSgrey}\rule{\textwidth}{1pt}}
		\put(0,55){\color{uniSgrey}\rule{\textwidth}{1pt}}
		\put(0,227){\color{uniSgrey}\rule{\textwidth}{1pt}}
		\put(0,272){\color{uniSgrey}\rule{\textwidth}{1pt}}
		
		% Hilfslinien für den Buchrücken, bei Bedarf einblenden
		% \put(\standardPageWidth,0){\rule{.4pt}{\paperheight}}
		% \put(\numexpr\spineWidth + \standardPageWidth\relax,0){\rule{.4pt}{\paperheight}}
		
		% Titel
		\doublespacing
		\put(\numexpr\standardPageWidth + \spineWidth + 20\relax, 55){% Position Text
			\makebox(0,172)[lc]{%
				% \begin{tabular}[t]{l{\baselineskip} l}
				\begin{tabular}{p{120mm}}  
					\textbf{{\LARGE \TitelDerArbeit}} \\ 
					\StudentVorname \: \StudentNachname \\
					\\ \\ 
					\TypDerArbeit\\
					\EndeMonat \: \EndeJahr
				\end{tabular}
		}}
		
	\end{picture}%
	\endgroup%
\end{document}