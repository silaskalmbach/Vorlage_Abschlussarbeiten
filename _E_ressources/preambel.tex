%------------------------------------------------------------------------------
%Description:  LaTeX Thesis Template preamble
%Author:       silas.kalmbach@ilek.uni-stuttgart.de
%Created:      2021-03-10
%------------------------------------------------------------------------------

% ============= Dokumentinformationen =============

\usepackage[ 
hidelinks
]{hyperref} 
\hypersetup{ 
	pdfauthor={\StudentVorname \StudentNachname}, 
	pdftitle={\TitelDerArbeit},
	pdfsubject={\TypDerArbeit}, 
	pdfkeywords={},
	colorlinks=true, %Einfärben statt der Box
	citecolor=blue,
	linkcolor=black,
	urlcolor=blue
%	linkbordercolor={1 1 1},
%	urlbordercolor={0 0.25 0.56},
%	citebordercolor={1 1 1},
%	menubordercolor={1 1 1},
%	filebordercolor={1 1 1},
%	citebordercolor={1 1 1}
}

% ============= Farbschema =============

\usepackage{color}      %Verwendung von Farben
\usepackage{xcolor}		%Definition eigener Farben

% Uni Stuttgart Colors
\definecolor{uniSlightblue}{RGB}{0,190,255}
\definecolor{uniSdarkblue}{RGB}{0,65,145}
\definecolor{uniSdarkgrey}{RGB}{62, 68, 76}
\definecolor{uniSgreyblue}{RGB}{125,198,234}
\definecolor{uniSgrey}{RGB}{159,153,152}

% Uni Stuttgart monochrom greyscale
\definecolor{uniSblack}{RGB}{0,0,0}
\definecolor{uniSdarkgrey}{RGB}{62,68,76}
\definecolor{uniSgrey}{RGB}{159,153,152}
\definecolor{uniSlightgrey}{RGB}{185,186,187}


% ============= Standart Packages und Einstellungen =============

\usepackage{ifluatex}
\ifluatex
	\usepackage[utf8]{luainputenc} %Schriftkodierung LuaLaTeX
\else
	\usepackage[utf8]{inputenc} %Schriftkodierung PDFLaTeX
\fi

\usepackage{fancyvrb} % Sophisticated verbatim text
\usepackage[english, ngerman]{babel} %Sprachen
\usepackage{graphicx} %Einbinden von Bildern
\usepackage{amsfonts} % zusätzliche Schriftzeichen der American Mathematical Society
\usepackage{amsmath,amssymb} % zusätzliche Schriftzeichen der American Mathematical Society
\usepackage{typearea} %Aufteilung von Rändern und Textbereich
\usepackage{nicefrac}% weitere Alternative mit mehreren Zeichen für Brüche
\usepackage{pdfpages} %Einfügen von PDFs
\usepackage{appendix} %Anhang
\usepackage{fvextra} %automatic line breaking and improved math mode
\usepackage{csquotes} %inline and display quotations
\usepackage[german, nameinlink]{cleveref} % Mehr Möglichkeiten zum referenzieren
\usepackage{gensymb} %Symbole: \degree \celsius \perthousand \ohm \micro
\tolerance = 400  % möglicher Abstand zwischen den Wörtern erhöhen
\usepackage[section]{placeins} %\FloatBarrier Verhindert das Abbildungen in nachfolgenden Kapitel rutschen
\makeatletter\AtBeginDocument{\expandafter\renewcommand\expandafter\subsection\expandafter{\expandafter\@fb@secFB\subsection}}\makeatother %FloatBarrier für Subsection
\usepackage{subcaption} % Bilder nebeneinander
\captionsetup[subfigure]{list=true, labelfont=bf, position=top}
\usepackage{wrapfig} % Bilder von TExt umgeben
\usepackage{dirtree} % Darstellen von Verzeichnisstrukturen
\usepackage{xstring} % Manipulieren von Strings


% ============= Programmierung (Visualisierung) =============

\if 1\mintedinstall
	\usepackage[cache=false]{minted} %Highlighting von Code, Benötigt Python und Pygments hinterlegt in den Umgebungsvariablen (pip install Pygments) ,cmd Eingabe->pdflatex -synctex=1 -interaction=nonstopmode -shell-escape %.tex
	\usemintedstyle{tango, xleftmargin=20pt, linenos, fontsize=\small, breaklines=true, tabsize=4} %borland, bw, tango, frame=leftline
\fi


\usepackage{listings}
%\lstset{
%	language=python,
%	otherkeywords={1, 2, 3, 4, 5, 6, 7, 8 ,9 , 0, -, =, +, [, ], (, ), \{, \}, :, *, !},
%}

\lstdefinestyle{UniStStyle}{
%	backgroundcolor=\color{uniSlightgrey},   
	commentstyle=\itshape\color{uniSlightblue}\lstset{columns=fullflexible},
	keywordstyle=\color{uniSdarkblue},
	numberstyle=\tiny\color{uniSlightgrey},
	stringstyle=\color{uniSgreyblue},
	basicstyle=\ttfamily\small,
	xleftmargin=20pt,
%	xrightmargin=3pt,
	breakatwhitespace=false,         
	breaklines=true,                 
	captionpos=b,                    
	keepspaces=true,                 
	numbers=left,                    
	numbersep=10pt,                  
	showspaces=false,                
	showstringspaces=false,
	showtabs=false,                  
	tabsize=4
}

\lstset{style=UniStStyle}


% ============= Tabelleneinstellungen =============

\usepackage{multirow}						% ermoeglicht mehrere Zeilen in einer Tabellenzeile
\usepackage{multicol} 						% mehre Spalten auf eine Seite

\usepackage{longtable}[=v4.13]				% alternatives Tabellenpackage inkl. longtabu und Farben
\usepackage{tabu}
\usepackage{float}							% um [H] bei float Objekten verwenden zu können
\usepackage{booktabs}						% um \toprule, \midrule und \bottomrule verwenden zu können
\usepackage{tabularx}
\usepackage{tabulary}
\usepackage{csvsimple} %Excel Improtieren als CSV 
\usepackage{siunitx} %Wissenschaftliche Angabe der Einheiten \num{number}
%für Excel to Latex Add-on
\usepackage{colortbl} %Farbige Tabellen
\usepackage{rotating} %Rotieren von Tabellen
\usepackage{bigstrut} %Für Tabellen mit Vertikalen Linien


% ============= Standart Einstellungen der Pfade =============

\graphicspath{{./_A_frontmatter/02_cover/images}{./_B_mainmatter/images/}} %Bilderordner


% ============= Schriftarten =============

\ifluatex
%% ============= Lua-LaTeX =============
% Prüfen ob Datei mit Schriftart UniversforUniS vorhanden:
    \IfFileExists{./_E_ressources/fonts/UniversforUniS55Rm-Regular.ttf}{
        % Wenn ja, diese Schriftart auswählen
    	\usepackage[no-math]{fontspec} 
    	\setmainfont{UniversforUniS}[
    	Extension      = .ttf,
    	Path           = _E_ressources/fonts/,
    	UprightFont    = *55Rm-Regular,
    	ItalicFont     = *45LtObl-Rg,
    	BoldFont       = *65Bd-Regular,
    	BoldItalicFont = *45LtObl-Rg,
    	]
    	\setsansfont{UniversforUniS}[
    	Extension      = .ttf,
    	Path           = _E_ressources/fonts/,
    	UprightFont    = *55Rm-Regular,
    	ItalicFont     = *45LtObl-Rg,
    	BoldFont       = *65Bd-Regular,
    	BoldItalicFont = *45LtObl-Rg,
    	]
    	%\setmonofont{}[]
    	\addtokomafont{disposition}{\rmfamily} %Überschriften selbe Schriftart wie Text
    	\renewcommand\familydefault\sfdefault % Formeln in Sans-Serif
    	\usepackage[italic, symbolgreek]{mathastext} % Formeln in Sans-Serif
    	\renewcommand\familydefault\rmdefault % Formeln in Sans-Serif
    }{
		\usepackage[no-math]{fontspec}
		\setmainfont{Arial}
		
		\addtokomafont{disposition}{\rmfamily} %Überschriften selbe Schriftart wie Text
		\renewcommand\familydefault\sfdefault % Formeln in Sans-Serif
		\usepackage[italic, symbolgreek]{mathastext} % Formeln in Sans-Serif
		\renewcommand\familydefault\rmdefault % Formeln in Sans-Serif
    }

\else
%% ============= PDF-LaTeX =============
	\usepackage[T1]{fontenc} %Richtige Trennung von Umlauten
	\usepackage[scale=1]{tgheros}
%	\usepackage{fourier}
	\renewcommand\familydefault\sfdefault % alles in Sans-Serif
	\usepackage[italic, symbolgreek]{mathastext} % Formeln in Sans-Serif
\fi



\usepackage{setspace} %Zeilenabstand
\setstretch{1.05} %Zeilenabstand

\setlength{\parindent}{0pt} %Wenn Absatzabstand, dann Einzug unnötig
\setlength{\parskip}{7pt} %Absatzabstand



% ============= Kopf- und Fußzeile =============

\usepackage[automark, headsepline, autooneside=false]{scrlayer-scrpage}

% Führe nur aus, wenn nicht article als document class festgelegt wurde (frontcover.tex)
\makeatletter
\@ifclassloaded{scrartcl}{}{     
    \automark[section]{chapter}	%Automatische Kapitelangabe in Kopfzeile
    \renewcommand*{\chaptermarkformat}{} %Kapitel ohne Kapitelnummer
}
\makeatother

\renewcommand*{\sectionmarkformat}{} %Kapitel ohne Kapitelnummer
\clearpairofpagestyles %aktuelle Einetsllungen aus Kopf- und Fußzeile löschen
\setkomafont{pageheadfoot}{\small}		%Font für Kopfzeile
\setkomafont{pagefoot}{\small}  		%Font für Fußzeile
\setkomafont{pagenumber}{\small} 	%Font für Seitennummer
\KOMAoptions{headsepline = true} %Kopzeile Linie

\lehead[]{\rightmark} %Zweiseitig
\cohead[]{}
\rohead[]{\leftmark\enskip \raisebox{0.7pt}{|}\enskip \thechapter} %Darstellung Kapitel

\lefoot[\pagemark]{\pagemark}
\lofoot[]{}
\cofoot[]{}
\rofoot[\pagemark]{\pagemark} %Seitenzahl


% ============= Literatur =============

\usepackage[									
    natbib=true,			% Naturwissenschaftlich
    backend=biber,			% Biber statt BibTex verwenden. In TexStudio einstellen!
    style=numeric,			% Stil im Verzeichnis alt. authoryear
    citestyle=numeric,	    % Stil im Fließtext
    giveninits=true,										
    doi=false,				% doi nicht mit angeben
    isbn=false,				% isbn nicht mit angeben
    sorting=none,			% Sortierung des Verzeichnis nach Autor, Titel, Jahr
    % sorting=none,			% Keine Sortierung des Verzeichnisses
]{biblatex}	

\addbibresource{_D_backmatter/literatur.bib} %Bibliographiedateien laden

\urlstyle{same}	% url Schriftart anpassen auf die des Dokuments anpassen
\usepackage{xpatch}% author in bold font in bibliography
\xpretobibmacro{author}{\mkbibbold\bgroup}{}{}
\xapptobibmacro{author}{\egroup}{}{}
\xpretobibmacro{bbx:shortauthor}{\mkbibbold\bgroup}{}{}
\xapptobibmacro{bbx:shortauthor}{\egroup}{}{}
\xpretobibmacro{bbx:editor}{\mkbibbold\bgroup}{}{}
\xapptobibmacro{bbx:editor}{\egroup}{}{}


% ============= Besondere Abkürzungen ============= 

\usepackage[figurename=Abb., % Beschriftung von Abbildung zu Abb.
tablename=Tab., % Beschriftung von Tabelle zu Tab.
labelfont=bf]{caption} %Label in Bold

% Führe nur aus, wenn nicht article als document class festgelegt wurde (frontcover.tex)
\makeatletter
\@ifclassloaded{scrartcl}{}{     
    \KOMAoptions{numbers = noenddot} % der Punkt hinter den Kapiteln etc wird entfernt: 1.1. wird zu 1.1
}
\makeatother


% ============= Daten Plotten ============= 

\usepackage{svg} 

\usepackage{tikz} %Grafiken Zeichnen
\if 0\tikzextern
	\usetikzlibrary{shapes, arrows, positioning, calc, chains, scopes}
\else
	\usepackage{shellesc} % Notwendig für LuaLaTeX
	\usetikzlibrary{shapes, arrows, positioning, calc, chains, scopes, external}
	\tikzexternalize[prefix=_E_ressources/tikz/,optimize command away=\includepdf] % activate and define tikz/ as cache folder
\fi

\usepackage{pgfplots}  %Kurven, Graphen
\usepgfplotslibrary{dateplot, fillbetween, groupplots}
\pgfkeys{/pgf/number format/.cd,fixed, use comma}
\pgfplotsset{
	compat=newest,
	every axis label/.append style={font={\rmfamily}},
}
\AtBeginEnvironment{tikzpicture}{\footnotesize} %Schriftart Tabelle \small, \footnotesize

\usepgfplotslibrary{groupplots}
\usepgfplotslibrary{dateplot}

\newlength{\figureheight} %Eigene Vatiable für Längen
\newlength{\figurewidth} %Eigene Vatiable für Längen
%
%% ============= Abkürzungen, Glossare ============= 
\usepackage[%
	nonumberlist, 			% keine Seitenzahlen anzeigen
	toc,          			% Einträge im Inhaltsverzeichnis
	section,    			% im Inhaltsverzeichnis auf section-Ebene erscheinen
%    xindy={language=german-duden,codepage=utf8}, % xindy zum Indexieren verwenden -> Perl.exe muss vorher installiert werden (https://www.activestate.com/products/perl/) Weitere Informationen unter http://tug.ctan.org/tex-archive/macros/latex/contrib/glossaries/glossariesbegin.html#x1-50011
    acronym,				% Separates Akronym-Verzeichnis
    nopostdot,				% Kein Punkt am Ende einer Beschreibung im Glossar
    automake=immediate
]{glossaries}

% Glossare & Abkürzungsverzeichnisse:
% Hier ergänzen sollten weitere Glossare erstellt werden

\newglossary[lug]{lat_upp}{lus}{luo}{Latin upper case letters}              % Big Latin Letters
\newglossary[llg]{lat_low}{lls}{llo}{Latin lower case letters}         		% Small Latin Letters
\newglossary[gug]{gre_upp}{gus}{guo}{Greek upper case letters}              % Big Greek Letters
\newglossary[glg]{gre_low}{gls}{glo}{Greek lower case letters}         		% Small Greek Letters         	
\newglossary[mog]{math_op}{mos}{moo}{Mathematical Operators}               	% Mathematical Operators

\makeglossaries

% Dateien die die Definitionen der Glossare enthalten
% Abbreviations

\newacronym{cent}{cent.}{century}
\newacronym{CFD}{CFD}{Computational Fluid Dynamics}

\newacronym{DFT}{DFT}{discrete Fourier transform}

\newacronym{FEM}{FEM}{Finite Element Method}
\newacronym{FV}{FV}{Finite Volume}

\newacronym{SVG}{SVG}{Scalable Vector Graphics}
% ------------------------
% Latin Lower Case Symbols
% ------------------------
% Dummy Symbol nicht löschen, dient der Formatierung des Abstandes zwischen Symbol und Erläuterung
\newglossaryentry{dummy_lat_low}{
	type=lat_low,
	sort=ZZZ,
	name={$~~~~~~~~~~~~~~$},
	description={$~~~~~~~$}
}
\newglossaryentry{u_i}{
	type=lat_low,
	sort=u_i,
	name={$u_{i}$},
	description={Velocity components in index notation}
}
\newglossaryentry{u_ref}{
	type=lat_low,
	sort=u_ref,
	name={$u_{ref}$},
	description={Reference velocity at $z_{ref}$}
}

% ------------------------
% Latin Upper Case Symbols
% ------------------------
% Dummy Symbol nicht löschen, dient der Formatierung des Abstandes zwischen Symbol und Erläuterung
\newglossaryentry{dummy_lat_upp}{
	type=lat_upp,
	sort=ZZZ,
	name={$~~~~~~~~~~~~~~$},
	description={$~~~~~~~$}
}
\newglossaryentry{Strouhal}{
	type=lat_upp,
	sort=St,
	name={$St$},
	description={Strouhal number}
}
\newglossaryentry{Iuu}{
	type=lat_upp,
	sort=Iuu,
	name={$I_{u_{i}u_{j}}$},
	description={Turbulence intensity}
}

% ------------------------
% Greek Upper Case Symbols
% ------------------------
% Dummy Symbol nicht löschen, dient der Formatierung des Abstandes zwischen Symbol und Erläuterung
\newglossaryentry{dummy_gre_upp}{
	type=gre_upp,
	sort=ZZZ,
	name={$~~~~~~~~~~~~~~$},
	description={$~~~~~~~$}
}
\newglossaryentry{dt}{
	type=gre_upp,
	sort=dt,
	name={$\Delta t$},
	description={Time step size}
}

% ------------------------
% Greek Lower Case Symbols
% ------------------------
% Dummy Symbol nicht löschen, dient der Formatierung des Abstandes zwischen Symbol und Erläuterung
\newglossaryentry{dummy_gre_low}{
	type=gre_low,
	sort=ZZZ,
	name={$~~~~~~~~~~~~~~$},
	description={$~~~~~~~$}
}
\newglossaryentry{kappa}{
	type=gre_low,
	sort=kappa,
	name={$\kappa$},
	description={Von-Kármán constant}
}
\newglossaryentry{rho}{
	type=gre_low,
	sort=alpha,
	name={$\rho_{u_{i}u_{j}}$},
	description={Normalized autocorrelation}
}

% ------------------------
% Mathematical Operators
% ------------------------
% Dummy Symbol nicht löschen, dient der Formatierung des Abstandes zwischen Symbol und Erläuterung
\newglossaryentry{dummy_math_op}{
	type=math_op,
	sort=ZZZ,
	name={$~~~~~~~~~~~~~~$},
	description={$~~~~~~~$}
}
\newglossaryentry{mean}{
	type=math_op,
	sort=2_mean,
	name={${\overline{(\:\:)}}$, ${(\:\:)_{mean}}$},
	description=Mean value
}
\newglossaryentry{std}{
	type=math_op,
	sort=3_std,
	name={${(\:\:)_{std}}$, $\sigma_{(\:\:)}$},
	description=Standard deviation
}

% Stil der Glossare
\setacronymstyle{short-long}

\newglossarystyle{abbreviation}{%
	\setglossarystyle{super}%
	\renewcommand{\glsgroupskip}{}					% Do not group items
	\renewcommand{\glsnamefont}[1]{\textbf{##1}}	% Make abbreviation bold
	\renewcommand*{\arraystretch}{1.0}				% Distance between entries
}

\newglossarystyle{symbols}{%
	\setglossarystyle{super}%
	\renewcommand{\glsgroupskip}{}								% Do not group items
	\renewcommand{\glsnamefont}[1]{\textbf{##1}}				% Make abbreviation bold
	\renewcommand*{\arraystretch}{1.0}							% Distance between entries
}


% Bei erstmaliger Verwendung Langform, anschließend Kurzform, immer in kursiv geschrieben
\defglsentryfmt{%
  \ifglsused{\glslabel}{%
    \glsgenentryfmt%
  }{%
    % Typeset first use
    \textit{\glsgenentryfmt}%
  }%
}

% Issue w/ glossaries: No Writes left
% See: https://tex.stackexchange.com/questions/289734/special-package-combination-gives-no-room-for-new-write
\usepackage{morewrites}

% ============= Testing ============= 

\usepackage{blindtext}	% Beispiel Texte einfügen per \blindtext
\usepackage{lipsum}

% ============= Formatierung Überschriften =============
% Führe folgendes nur aus, wenn nicht article als document class festgelegt wurde (frontcover.tex)
\makeatletter
\@ifclassloaded{scrartcl}{}{%
    % Change chapter fontsize
    \setkomafont{chapter}{\LARGE}
    
    % Change section fontsize
    \setkomafont{section}{\Large}
    
    % Set the space before and after chapters, sections, subsections
    \RedeclareSectionCommand[
      %runin=false,
      afterindent=false,
      beforeskip=0pt,
      afterskip=.1\baselineskip]{chapter}
      
    \RedeclareSectionCommand[
      %runin=false,
      afterindent=false,
      beforeskip=1\baselineskip,
      afterskip=.1\baselineskip]{section}
      
    \RedeclareSectionCommand[
      %runin=false,
      afterindent=false,
      beforeskip=.75\baselineskip,
      afterskip=.1\baselineskip]{subsection}
    
    % ============= Formatierung Inhaltsverzeichnis =============  
\if 1\tocstyle    
    % Format des TOC
    % Titelleiste "Inhaltsverzeichnis"
    \BeforeTOCHead[toc]{%
      \KOMAoptions{parskip=true}% no parskip in ToC
      \RedeclareSectionCommand[afterskip=1\baselineskip]{chapter}% skip after ToC title
    }
    
    % Zeilenabstände im TOC
    \DeclareTOCStyleEntry[beforeskip=1\baselineskip]{chapter}{chapter}
    
    % Kein Einzug vor Unterkapiteln, H-Space zwischen Nummern und Titel automatisch anpassen
    \RedeclareSectionCommands[tocindent=0em,tocdynnumwidth]{%
        chapter,section
    }
    
    % Keine punktierten Linien zwischen section und pagenumber
    \RedeclareSectionCommand[
      toclinefill=\hfill
    ]{section}
    
    % Keine punktierten Linien im Abbildungs-/Tabellenverzeichnis
    \DeclareTOCStyleEntry[linefill=\hfill]{default}{figure}
    \DeclareTOCStyleEntry[linefill=\hfill]{default}{table}
\fi
}
\makeatother
