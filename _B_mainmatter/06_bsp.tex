\chapter{Erweiterte Formatierung}

\section{Float Objekte}

\begin{description}
	\item[h:] an der Stelle, an der es in der Eingabedatei angegeben ist (here)
	\item[t:] am oberen Ende der aktuellen oder Folgeseite (top)
	\item[b:] am unteren Ende der aktuellen Seite (bottom)
	\item[p:] auf einer eigenen Seite für ein oder mehrere Gleitobjekte (page)
	\item[!:] Überschreiben Sie die internen Parameter, die LaTeX zur Bestimmung "guter" Gleitkommapositionen verwendet.
	\item[H:] Setzt den Float an genau die Stelle im LaTeX-Code. Erfordert das float-Paket.
\end{description}

\section{Einheiten}
Bei der Verwendung von Einheiten wird in der Regel bei Wissenschaftlichen Arbeiten ein schmales Leerzeichen verwendet.\\
1 m : Leerzeichen \\
1\,m : schmales Leerzeichen \\