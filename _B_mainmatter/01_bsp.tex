\chapter{Grundlagen}
\label{ch:01_grundlagen}

\section{Was ist \LaTeX}
\label{sec:01_was_ist_latex}

\LaTeX ist ein Layout- und Satzprogramm für wissenschaftliches Arbeiten. Es basiert auf dem Satzprogramm \TeX\, das von Donald Knuth von der Stanford University entwickelt wurde (seine erste Version erschien 1978).

\section{Textformatierung}
\label{sec:01_textformatierung}

Text kann unter anderem \textit{kursiv}, \textbf{fett} oder \underline{unterstrichen} formatiert werden. Mathematische Zeichen und Formeln können durch spezielle Befehle in \LaTeX erzeugt werden, bspw. $\sigma=E\cdot\varepsilon$.

\section{Gliederung}
\label{sec:01_gliederung}

Im wesentlichen wird das Dokument untergliedert in Kapitel (\verb!\chapter{}!), Abschnitte (\verb!\section{}!) und Unterabschnitte (\verb!\subsection{}!). Auch eine weitere Untergliederung ist möglich. Ab der Ebene (\verb!\subsection{}!) werden Abschnitte in dieser Vorlage im Inhaltsverzeichnis aus Gründen der Übersichtlichkeit nicht abgebildet.

\section{Verweise}
\label{sec:01_verweise}

% Prefixes
% ch: 	    chapter
% sec: 	    section
% subsec: 	subsection
% fig: 	    figure
% tab: 	    table
% eq: 	    equation
% lst: 	    code listing
% itm: 	    enumerated list item
% alg: 	    algorithm
% app: 	    appendix subsection 

Auf Kapitel und Abschnitte kann verwiesen werden. Beispielweise beinhaltet \cref{ch:01_grundlagen} die \crefrange{sec:01_was_ist_latex}{sec:01_zitation}.

Neben Kapiteln und Abschnitten kann auf Bilder, Tabellen, Gleichungen etc. verwiesen werden. Diese müssen im \LaTeX-Code mit \verb!\label{<prefix>:<label>}! versehen werden. 
\section{Zitation}
\label{sec:01_zitation}

Zitate können unter anderem wie folgt eingefügt werden:

\begin{itemize}

    \item Indirektes Zitat, eine Quelle: \cite{yang_application_2003}. 
    
    \item Indirektes Zitat, mehrere Quellen \cite{yang_application_2003, kroll_computational_2016}
    
    \item Direktes Zitat: Wie bereits \citet[][p.97]{yang_application_2003} sagte, \enquote{\lipsum[1][1]}
    
    \item \citet[][p.365]{yang_application_2003} stellte fest: \blockquote{\lipsum[1][2]}
    
\end{itemize}