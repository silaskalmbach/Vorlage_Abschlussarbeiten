\chapter{Einfügen von Tabellen}
\label{chap:tabellen}

\section{Beispieltabelle}

\begin{table}[h]
	\caption{Beispieltabelle}
	\label{tab:Parameter}
	\centering
	\begin{tabularx}{\textwidth}{p{0.32\textwidth}p{0.32\textwidth}p{0.32\textwidth}}
																					\toprule
		Eins                       	     & Zwei    			& Drei 				\\ 	\midrule
		Vier                       	     & Fünf    			& Sechs				\\
		Sieben                           & Acht         	& Neun				\\ 	\bottomrule	
	\end{tabularx}
\end{table}

\begin{longtabu} to \textwidth {@{} X X[c] X[r] @{}}
	\caption{Tabelle auf Textbreite mit drei gleich großen Spalten}\label{tab:bsp1} \\
	\toprule
	Spalte 1 linksbündig & Spalte 2 zentriert & Spalte 3 rechtsbündig \\
	\midrule
1 , 2                           & 3 , 2                           & 1 , 3 \\
2 , 4                           & 6 , 4                           & 2 , 6 \\
3 , 6                           & 9 , 6                           & 3 , 9 \\
	\bottomrule
\end{longtabu}

\begin{longtabu} to 140mm {@{} X X[c] X[r] @{}}
	\caption{Tabelle auf Textbreite mit drei gleich großen Spalten}\label{tab:bsp2} \\
	\toprule
	Spalte 1 linksbündig & Spalte 2 zentriert & Spalte 3 rechtsbündig \\
	\midrule
1 , 2                           & 3 , 2                           & 1 , 3 \\
2 , 4                           & 6 , 4                           & 2 , 6 \\
3 , 6                           & 9 , 6                           & 3 , 9 \\
	\bottomrule
\end{longtabu}
%

\begin{longtabu} to \textwidth {@{} X[1,l] X[1,c] X[1,r] @{}}
	%----- Kopfzeile erste Tabelle ----- %
	\caption{Tabelle über mehrere Seiten}\label{tab:mehrere Seiten} \\
	\toprule
	Spalte 1 linksbündig & Spalte 2 zentriert & Spalte 3 rechtsbündig \\
	\midrule
	\endfirsthead
	%----- Kopfzeile zweite Tabelle ----- %	
	\caption*{\textbf{Fortsetzung:} \cref{tab:mehrere Seiten}} \\
	\toprule
	Spalte 1 linksbündig & Spalte 2 zentriert & Spalte 3 rechtsbündig \\
	\midrule
	\endhead
	%----- Tabellenende erste Tabelle ----- %	
	\bottomrule
	\endfoot
	%----- Tabellenende zweite Tabelle ----- %
	\bottomrule    
	\endlastfoot
	%----- Inhalt der Tabelle Tabelle ----- %	
1 , 2                           & 3 , 2                           & 1 , 3 \\
2 , 4                           & 6 , 4                           & 2 , 6 \\
3 , 6                           & 9 , 6                           & 3 , 9 \\
4 , 8                           & 12 , 8                           & 4 , 12 \\
5 , 10                           & 15 , 10                           & 5 , 15 \\
6 , 12                           & 18 , 12                           & 6 , 18 \\
7 , 14                           & 21 , 14                           & 7 , 21 \\
8 , 16                           & 24 , 16                           & 8 , 24 \\
9 , 18                           & 27 , 18                           & 9 , 27 \\
10 , 20                           & 30 , 20                           & 10 , 30 \\
11 , 22                           & 33 , 22                           & 11 , 33 \\
12 , 24                           & 36 , 24                           & 12 , 36 \\
13 , 26                           & 39 , 26                           & 13 , 39 \\
14 , 28                           & 42 , 28                           & 14 , 42 \\
15 , 30                           & 45 , 30                           & 15 , 45 \\
16 , 32                           & 48 , 32                           & 16 , 48 \\
17 , 34                           & 51 , 34                           & 17 , 51 \\
18 , 36                           & 54 , 36                           & 18 , 54 \\
19 , 38                           & 57 , 38                           & 19 , 57 \\
\end{longtabu}