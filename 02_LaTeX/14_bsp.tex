\chapter{tikz - Grafiken}
\label{sec:tikz}

\section{Beispielkapitel tikz - Grafiken}
\label{sec:beispiel}

\blindtext

\begin{figure}[!htbp]
\centering
\begin{align*}
f(x)=\tanh(x)
\end{align*}
\begin{tikzpicture}
\begin{axis}
[
width=200pt,
height=200pt,
axis x line=middle, xmin=-4, xmax=4, xtick={-3,...,3}, xlabel=$x$,
axis y line=middle, ymin=-1.5, ymax=1.5, ytick={-1,...,1}, ylabel=$f(x)$,
scale only axis=true,
samples=101
]
\addplot[blue,mark=none, very thick]{tanh(x)};
%		\legend{$\tanh(x)$}
\end{axis}
\end{tikzpicture}
\caption[Tangens hyperbolicus Aktivierungsfunktion]{Tangens hyperbolicus}
\label{fig:AktivTan}
\end{figure}

\blindtext


\begin{figure}[!htbp]
	\centering
	\begin{tikzpicture}[]
	\tikzstyle{netnode}=[circle, inner sep=0pt, text width=22pt, align=center, line width=1.0pt]
	\tikzstyle{inputnode}=[netnode, fill=lightgray,draw=black]
	\tikzstyle{hiddennode}=[netnode, fill=lightgray,draw=black]
	\tikzstyle{outputnode}=[netnode, fill=lightgray,draw=black]
	\tikzstyle{signal}=[arrows={-latex},draw=black,line width=1pt,rounded corners=4pt]
	
	\def\nodedist{35pt}
	\def\layerdist{80pt}
	\def\pindist{20pt}
	
	\tikzstyle{every pin edge}=[signal]
	\tikzstyle{annot} = [text width=6em, text centered]
	
	\foreach \y in {1,...,3}
	\node[inputnode, pin={[pin edge={latex-}, pin distance=\pindist]left:Eingabe \y}] 
	(I\y) at (0,-\y*\nodedist) {$i_\y$};  
	
	\foreach \y in {1,...,4}
	\node[hiddennode] 
	(H1\y) at ($(\layerdist,-\y*\nodedist) +(0, 0.5*\nodedist)$) {$h_\y^1$};
	
	\foreach \y in {1,...,4}
	\node[hiddennode] 
	(H2\y) at ($(2*\layerdist,-\y*\nodedist) +(0, 0.5*\nodedist)$) {$h_\y^2$};
	
	\foreach \y in {1,...,2}
	\node[outputnode, pin={[pin edge={-latex}, pin distance=\pindist]right:Ausgabe \y}]
	(O\y) at ($(H21) + (\layerdist, -\y*\nodedist)$) {$o_\y$};
	
	\foreach \dest in {1,...,4}
	\foreach \source in {1,...,3}
	\draw[signal] (I\source) -- (H1\dest);
	
	\foreach \dest in {1,...,4}
	\foreach \source in {1,...,4}
	\draw[signal] (H1\source) -- (H2\dest);
	
	\foreach \dest in {1,...,2}
	\foreach \source in {1,...,4}
	\draw[signal] (H2\source) edge (O\dest);
	
	\node[annot, above=4pt of H11] (hl) {verborgene Schicht 1};
	\node[annot, above=4pt of H21] (hl) {verborgene Schicht 2};
	\node[annot] at (I1 |- hl) {Eingabe\-schicht};
	\node[annot] at (O1 |- hl) {Ausgabe\-schicht};
	\end{tikzpicture}
	\caption{Schematischer Aufbau eines künstlichen neuronalen Netzes \cite[Abb. nach][]{Frochte.2019}}
	\label{fig:DNN}
\end{figure}


\section{Beispielkapitel Standard Grafik}
\label{sec:graf}

\blindtext

\begin{figure}[h]
  \centering  
  	\includegraphics[scale=0.5]{Images/ILEK-logo.jpg}
  \caption{ILEK Logo}
  \label{fig:starwars}
\end{figure}

\blindtext

