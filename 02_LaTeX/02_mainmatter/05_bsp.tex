\chapter{tikz - Grafiken}
\label{sec:tikz}

\section{Beispielkapitel tikz - Grafiken}
\label{sec:beispiel}

\subsection{Darstellung von Funktionen}

\begin{figure}[htb]
\centering
\begin{align*}
f(x)=\tanh(x)
\end{align*}
\begin{tikzpicture}
\begin{axis}
[
width=200pt,
height=200pt,
axis x line=middle, xmin=-4, xmax=4, xtick={-3,...,3}, xlabel=$x$,
axis y line=middle, ymin=-1.5, ymax=1.5, ytick={-1,...,1}, ylabel=$f(x)$,
scale only axis=true,
samples=101
]
\addplot[uniSdarkblue, mark=none, very thick]{tanh(x)};
%		\legend{$\tanh(x)$}
\end{axis}
\end{tikzpicture}
\caption[Tangens hyperbolicus Aktivierungsfunktion]{Tangens hyperbolicus}
\label{fig:AktivTan}
\end{figure}

\newpage
\subsection{for-Schleifen bei der Grafikerzeugung}

\begin{figure}[htb]
	\centering
	\begin{tikzpicture}[]
	\tikzstyle{netnode}=[circle, inner sep=0pt, text width=22pt, align=center, line width=1.0pt]
	\tikzstyle{inputnode}=[netnode, fill=uniSlightgrey,draw=black]
	\tikzstyle{hiddennode}=[netnode, fill=uniSlightgrey,draw=black]
	\tikzstyle{outputnode}=[netnode, fill=uniSlightgrey,draw=black]
	\tikzstyle{signal}=[arrows={-latex},draw=uniSblack,line width=1pt,rounded corners=4pt]
	
	\def\nodedist{35pt}
	\def\layerdist{80pt}
	\def\pindist{20pt}
	
	\tikzstyle{every pin edge}=[signal]
	\tikzstyle{annot} = [text width=6em, text centered]
	
	\foreach \y in {1,...,3}
	\node[inputnode, pin={[pin edge={latex-}, pin distance=\pindist]left:Eingabe \y}] 
	(I\y) at (0,-\y*\nodedist) {$i_\y$};  
	
	\foreach \y in {1,...,4}
	\node[hiddennode] 
	(H1\y) at ($(\layerdist,-\y*\nodedist) +(0, 0.5*\nodedist)$) {$h_\y^1$};
	
	\foreach \y in {1,...,4}
	\node[hiddennode] 
	(H2\y) at ($(2*\layerdist,-\y*\nodedist) +(0, 0.5*\nodedist)$) {$h_\y^2$};
	
	\foreach \y in {1,...,2}
	\node[outputnode, pin={[pin edge={-latex}, pin distance=\pindist]right:Ausgabe \y}]
	(O\y) at ($(H21) + (\layerdist, -\y*\nodedist)$) {$o_\y$};
	
	\foreach \dest in {1,...,4}
	\foreach \source in {1,...,3}
	\draw[signal] (I\source) -- (H1\dest);
	
	\foreach \dest in {1,...,4}
	\foreach \source in {1,...,4}
	\draw[signal] (H1\source) -- (H2\dest);
	
	\foreach \dest in {1,...,2}
	\foreach \source in {1,...,4}
	\draw[signal] (H2\source) edge (O\dest);
	
	\node[annot, above=4pt of H11] (hl) {verborgene Schicht 1};
	\node[annot, above=4pt of H21] (hl) {verborgene Schicht 2};
	\node[annot] at (I1 |- hl) {Eingabe\-schicht};
	\node[annot] at (O1 |- hl) {Ausgabe\-schicht};
	\end{tikzpicture}
	\caption[Schematischer Aufbau eines künstlichen neuronalen Netzes] %hier kann ein alternativer/ verkürzter Text für das Abbildungsverzeichnis angegeben werden
	{Schematischer Aufbau eines künstlichen neuronalen Netzes \cite[Abb. nach][]{Frochte.2019}} %Bildunteschrift
	\label{fig:DNN}
\end{figure}


\subsection{Einbeziehung von Daten aus CSV-Datei}

\setlength{\figureheight}{5cm}
\setlength{\figurewidth}{\textwidth}
\begin{figure}[htbp]
	\centering
	\begin{tikzpicture}
		\begin{axis}[
			width=\figurewidth,
			height=\figureheight,
			date coordinates in = x,
			xmin=2018-01-01 00:00,
			xmax=2018-01-08 00:00,
			minor x tick num=1,
			xtick={2018-01-01 12:00, 2018-01-02 12:00, 2018-01-03 12:00, 2018-01-04 12:00, 2018-01-05 12:00, 2018-01-06 12:00, 2018-01-07 12:00},
			set layers,cell picture=true,
			xticklabel={\day.\month},
			xlabel={2018},
			ymin=-0.1, ymax=1.1, ytick={0,0.5,1},
			no marks,
			scaled y ticks = false,
			ylabel={[\nicefrac{\%}{100}]}
			,]
			\addplot+[color=black ] table[x=Zeit, y=Shadestep, col sep=comma, skip first n=0] {./Data/Example.csv};
			\legend{Shadestep};
		\end{axis}
	\end{tikzpicture}
	\caption[Photometrische Regelung der adaptiven Verglasung im Juli 2018, nach 500 Episoden]{Photometrische Regelung der adaptiven Verglasung nach 500 Episoden, für den Zeitraum vom 01. bis 07. Juli 2018} \label{fig:AVPhJun500}	
\end{figure}

\newpage
\subsection{Einbeziehung von Daten aus CSV-Datei und Gruppierung von Grafiken}

\begin{figure}[htb]
	\setlength{\figureheight}{5cm}
	\setlength{\figurewidth}{\textwidth}
	\centering
	\begin{tikzpicture}
		\begin{groupplot}[
			width=\figurewidth,
			height=\figureheight,
			date coordinates in = x,
			xmin=2018-01-01 00:00,
			xmax=2018-01-08 00:00,
			minor x tick num=1,
			xtick={2018-01-01 12:00, 2018-01-02 12:00, 2018-01-03 12:00, 2018-01-04 12:00, 2018-01-05 12:00, 2018-01-06 12:00, 2018-01-07 12:00},
			set layers,cell picture=true,
			xticklabel={\day.\month},
			xlabel={2018},
			no marks,
			scaled y ticks = false,
			group style = {group size = 1 by 7,
				horizontal sep =1cm,
				vertical sep = 0cm,},]
			
			\nextgroupplot[xlabel={},xticklabels={},ylabel={[°C]}]
			\addplot[color=black ] table[x=Zeit, y=T01, col sep=comma, skip first n=0] {./Data/Example.csv};
			\legend{T01};
			
			\nextgroupplot[xlabel={},xticklabels={},ylabel={[klx]}]
			\addplot[color=black ] table[x=Zeit, y expr=\thisrow{B01}/1000, col sep=comma, skip first n=0] {./Data/Example.csv};
			\legend{B01};
			
			\nextgroupplot[ylabel={[kW]}, ymin=-0.02, ymax=0.32, ytick={0,0.1,0.2,0.3},]
			\addplot+[color=black , name path=A] table[x=Zeit, y expr=\thisrow{QBeleuchtung}+\thisrow{QHeizung}+\thisrow{QKuehlung}, col sep=comma, skip first n=0] {./Data/Example.csv};
			\addplot+[draw=none,name path=B, mark=none] table[x=Zeit, y expr=0, col sep=comma, skip first n=0] {./Data/Example.csv}; 
			\addplot+[gray, fill opacity=0.1] fill between[of=A and B];
			\legend{Q\textsubscript{HEAT}+Q\textsubscript{COOL}+Q\textsubscript{LIGHT}};

		\end{groupplot}
	\end{tikzpicture}
	\caption[Kombinierte Regelung im Januar 2018, nach 500 Episoden]{Kombinierte Regelung nach 500 Episoden, für den Zeitraum vom 01. bis 07. Januar 2018} \label{fig:KomJan18}	
\end{figure}


\section{Beispielkapitel Standard Grafik}
\label{sec:graf}

\subsection{Einfaches Bild}

\begin{figure}[H]
  \centering  
  	\includegraphics[scale=0.5]{Images/ILEK-logo.jpg}
  \caption{ILEK Logo}
  \label{fig:starwars}
\end{figure}

\subsection{Gruppierung von Bildern}

\begin{figure}[H]
	\centering
	\begin{subfigure}[b]{0.3\textwidth}
		\centering
		\includegraphics[width=\textwidth]{Images/ILEK-logo.jpg}
		\caption{$y=x$}
		\label{fig:a}
	\end{subfigure}
%	\hfill
\hspace{1cm}
	\begin{subfigure}[b]{0.3\textwidth}
		\centering
		\includegraphics[width=\textwidth]{Images/ILEK-logo.jpg}
		\caption{$y=3sinx$}
		\label{fig:b}
	\end{subfigure}
	\\ 
	\begin{subfigure}[b]{0.3\textwidth}
		\centering
		\includegraphics[width=\textwidth]{Images/ILEK-logo.jpg}
		\caption{$y=3sinx$}
		\label{fig:c}
	\end{subfigure}
%	\hfill
\hspace{1cm}
	\begin{subfigure}[b]{0.3\textwidth}
		\centering
		\includegraphics[width=\textwidth]{Images/ILEK-logo.jpg}
		\caption{$y=5/x$}
		\label{fig:five over x}
	\end{subfigure}
	\caption{Vier Bilder}
	\label{fig:d}
\end{figure}

\subsection{Bilder und Tabellen im Fließtext}

\begin{wraptable}[]{o}[0cm]{9cm}
	\begin{center}
		\begin{tabular}{|l|l|l|}
			\hline
			r & R & right side of the text\\
			l & L & left side of the text\\
			i & I & inside edge–near the binding\\
			& &  (in a twoside document)\\
			o & O & outside edge–far from the binding\\
			\hline
		\end{tabular}
	\end{center}
	\caption{The uppercase version allows the figure to float. The lowercase version means exactly here.}%\citep{cite}
\end{wraptable}

\blindtext[2]

%\begin{wrapfigure}[Zeile]{Position}[Randüberhang]{Breite}
\begin{wrapfigure}[]{o}[0cm]{6.5cm}
	\includegraphics[width=6cm,angle=0]{bild}
	\caption{Bildbezeichnung}
	\label{fig:bild}
\end{wrapfigure}

\blindtext[2]