% ============= Angaben zur Arbeit =============

\def\mode{1} % 1 = onepage; 2=twopage (Ein- oder Zweiseitiges Dokument)
\def\tikzextern{0} % 0 = Nein; 1 = Ja (TikZ Grafiken Extern Speicehrn)

\newcommand{\titel}{Xxxxx Xxxxx Xxxxx Xxxxx Xxxxx Xxxxx Xxxxx Xxxxx Xxxxx Xxxxx Xxxxx} % Hier den Titel der Arbeit angeben 
\newcommand{\arbeit}{MASTERARBEIT} % Hier die Art der Abschlussarbeit eingeben in Großbuchstaben 
\newcommand{\Nummer}{XX} % Nummerrierung der Arbeit (mit Berteurer zu klären)
\newcommand{\EndeMonatTxt}{Xxxxx} % Angabe des Monates der Abgabe als Wort
\newcommand{\EndeJahr}{20XX} % Angabe des Jahres der Abgabe
\newcommand{\bereuer}{Xxxxx Xxxxx} % Hier den Betreuer der Arbeit angeben
\newcommand{\pruefer}{Xxxxx Xxxxx} % Hier den Prüfer der Arbeit angeben 
\newcommand{\student}{Xxxxx Xxxxx} %Hier den Studierendenname der Arbeit angeben 
\newcommand{\profEins}{Prof. Dr.-Ing. M.Arch. Lucio Blandini} % Hier den Professor1 des Instituts angeben 
\newcommand{\profZwei}{Prof. Dr.-Ing. Dr.-Ing. E.h. Dr. h.c. Werner Sobek} % Hier den Professor2 des Instituts angeben 
\newcommand{\profDrei}{Prof. Dr.-Ing. Balthasar Nov\'{a}k} % Hier den Professor3 des Instituts angeben 
\newcommand{\institut}{Institut f{\"u}r Leichtbau Entwerfen und Konstruieren} % Hier das Institut angeben
\newcommand{\titelbild}{Images/Titelseite/Titelbild} % Hier den Pfad des Titelbildes eingeben


% ============= Pakete importieren =============

%Bei Fragen und Anregungen zu dieser Vorlage können Sie sich gerene bei Silas Kalmbach (silas.kalmbach@ilek.uni-stuttgart.de) melden.

% ============= Einseitig =============
\documentclass[a4paper,12pt]{scrreprt}
\usepackage[left= 2.5cm, right = 2cm, top=2.5cm]{geometry}

% ============= Doppelseitig =============
%\documentclass[a4paper,12pt, openright, twoside]{scrreprt}
%\usepackage[left= 2.5cm, right = 2cm, top=2.5cm]{geometry}
%% -> Anpassungen an die Kopf- und Fußzeilen beachten!


% ============= Dokumentinformationen =============

\usepackage[ 
hidelinks
]{hyperref} 
\hypersetup{ 
	pdfauthor={\student}, 
	pdftitle={\titel},
	pdfsubject={\arbeit}, 
	pdfkeywords={}
}


% ============= Farbschema =============

\usepackage{color}      %Verwendung von Farben
\usepackage{xcolor}		%Definition eigener Farben

% Uni Stuttgart Colors
\definecolor{uniS_lightblue}{RGB}{0,190,255}
\definecolor{uniS_darkblue}{RGB}{0,65,145}
\definecolor{uniS_darkgrey}{RGB}{62, 68, 76}
\definecolor{uniS_greyblue}{RGB}{125,198,234}
\definecolor{uniS_grey}{RGB}{159,153,152}

% Uni Stuttgart monochrom greyscale
\definecolor{uniS_black}{RGB}{0,0,0}
\definecolor{uniS_darkgrey}{RGB}{62,68,76}
\definecolor{uniS_grey}{RGB}{159,153,152}
\definecolor{uniS_lightgrey}{RGB}{185,186,187}


% ============= Standart Packages und Einstellungen =============

%\usepackage[utf8]{inputenc} %Schriftkodierung PDFLaTeX
\usepackage[utf8]{luainputenc} %Schriftkodierung LuaLaTeX
\usepackage[english, ngerman]{babel} %Sprachen
\usepackage{graphicx} %Einbinden von Bildern
\usepackage{amsfonts} % zusätzliche Schriftzeichen der American Mathematical Society
\usepackage{amsmath,amssymb} % zusätzliche Schriftzeichen der American Mathematical Society
\usepackage{typearea} %Aufteilung von Rändern und Textbereich
%\usepackage{minted} %Highlighting von Code, Benötigt Python ,cmd Eingabe->pdflatex -synctex=1 -interaction=nonstopmode --shell-escape %.tex
\usepackage{tikz} %Grafiken Zeichnen
\usetikzlibrary{shapes,arrows,positioning,calc,chains,scopes}
\usepackage{pgfplots}  %Kurven, Graphen
%\usepgfplotslibrary{dateplot, fillbetween, groupplots}
\usepackage{pdfpages} %Einfügen von PDFs
\usepackage{appendix} %Anhang
\usepackage{csquotes}
\usepackage[german]{cleveref} % Mehr Möglichkeiten zum referenzieren
\usepackage{gensymb} %Symbole: \degree \celsius \perthousand \ohm \micro
\tolerance = 400  % möglicher Abstand zwischen den Wörtern erhöhen
\usepackage[section]{placeins} %\FloatBarrier Verhindert das Abbildungen in nachfolgenden Kapitel rutschen
\makeatletter\AtBeginDocument{\expandafter\renewcommand\expandafter\subsection\expandafter{\expandafter\@fb@secFB\subsection}}\makeatother %FloatBarrier für Subsection
\usepackage{subcaption} % Bilder nebeneinander
\captionsetup[subfigure]{list=true, labelfont=bf, position=top}

% ============= Tabelleneinstellungen =============

\usepackage{multirow}						% ermoeglicht mehrere Zeilen in einer Tabellenzeile
\usepackage{multicol} 						% mehre Spalten auf eine Seite
\usepackage{longtable, tabu}				% alternatives Tabellenpackage inkl. longtabu und Farben
\usepackage{float}							% um [H] bei float Objekten verwenden zu können
\usepackage{booktabs}						% um \toprule, \midrule und \bottomrule verwenden zu können
\usepackage{tabularx}
\usepackage{csvsimple} %Excel Improtieren als CSV 
\usepackage{siunitx} %Wissenschaftliche Angabe der Einheiten \num{number}
%für Excel to Latex Add-on
\usepackage{colortbl} %Farbige Tabellen
\usepackage{rotating} %Rotieren von Tabellen
\usepackage{bigstrut} %Für Tabellen mit Vertikalen Linien


% ============= Standart Einstellungen der Pfade =============

\graphicspath{{./Images/}{./Images/Titelseite/}} %Bilderordner


% ============= Schriftarten =============
%PDF-LaTeX
%\usepackage[T1]{fontenc} %Richtige Trennung von Umlauten
%\usepackage[scale=0.87]{tgheros}
%\usepackage{fourier}

%Lua-LaTeX
\usepackage{fontspec}

\setmainfont{UniversforUniS}[
Extension      = .ttf,
Path           = fonts/,
UprightFont    = *55Rm-Regular,
ItalicFont     = *45LtObl-Rg,
BoldFont       = *65Bd-Regular,
BoldItalicFont = *45LtObl-Rg,
]

\setsansfont{UniversforUniS}[
Extension      = .ttf,
Path           = fonts/,
UprightFont    = *55Rm-Regular,
ItalicFont     = *45LtObl-Rg,
BoldFont       = *65Bd-Regular,
BoldItalicFont = *45LtObl-Rg,
]

%\setmonofont{CMU Typewriter Text}

\addtokomafont{disposition}{\rmfamily} %Überschriften selbe Schriftart wie Text

\usepackage{setspace} %Zeilenabstand
\setstretch{1.05} %Zeilenabstand

\setlength{\parindent}{0pt} %Wenn Absatzabstand, dann Einzug unnötig
\setlength{\parskip}{7pt} %Absatzabstand

\setkomafont{captionlabel}{\sffamily\bfseries} %Schrift der Labels
\setkomafont{caption}{\sffamily} %Schrift der Labelbezeichnung
\usepackage[eulergreek]{sansmath} %Schrift in Diagrammen
\AtBeginEnvironment{longtabu}{\sffamily} %Schriftart Tabelle %\usepackage{etoolbox}
\AtBeginEnvironment{figure}{\sffamily} %Schriftart Figure
\AtBeginEnvironment{tabu}{\sffamily} %Schriftart Tabelle
\AtBeginEnvironment{tikzpicture}{\footnotesize\sffamily} %Schriftart Tabelle \small, \footnotesize


% ============= Verzeichnisse =============

\RedeclareSectionCommands[
tocentryformat=\sffamily,
tocpagenumberformat=\sffamily
]{part,section,subsection,subsubsection,paragraph,subparagraph}

\DeclareTOCStyleEntry[pagenumberformat=\sffamily, entryformat=\sffamily]{tocline}{table}
\DeclareTOCStyleEntry[pagenumberformat=\sffamily, entryformat=\sffamily]{tocline}{figure}


% ============= Kopf- und Fußzeile >>Einseitig<< =============

\usepackage{scrlayer-scrpage}
\automark{chapter}	%Automatische Kapitelangabe in Kopfzeile
\renewcommand*{\chaptermarkformat}{} %Kapitel ohne Kapitelnummer
\clearpairofpagestyles %aktuelle Einetsllungen aus Kopf- und Fußzeile löschen
\setkomafont{pageheadfoot}{\small\sffamily}		%Font für Kopfzeile
\setkomafont{pagefoot}{\small\sffamily}  		%Font für Fußzeile
\setkomafont{pagenumber}{\small\sffamily} 	%Font für Seitennummer
\KOMAoptions{headsepline = true} %Kopzeile Linie

\lohead[]{}
\cohead[]{}
\rohead[]{\headmark\enskip \raisebox{0.7pt}{|}\enskip \thechapter} %Darstellung Kapitel

\lofoot[]{}
\cofoot[]{}
\rofoot[\pagemark]{\pagemark} %Seitenzahl


% ============= Kopf- und Fußzeile >>Doppelseitig<< =============

%\usepackage[automark, headsepline, autooneside=false]{scrlayer-scrpage}
%\automark[section]{chapter}	%Automatische Kapitelangabe in Kopfzeile
%\renewcommand*{\chaptermarkformat}{} %Kapitel ohne Kapitelnummer
%\renewcommand*{\sectionmarkformat}{} %Kapitel ohne Kapitelnummer
%\clearpairofpagestyles %aktuelle Einetsllungen aus Kopf- und Fußzeile löschen
%\setkomafont{pageheadfoot}{\small\sffamily}		%Font für Kopfzeile
%\setkomafont{pagefoot}{\small\sffamily}  		%Font für Fußzeile
%\setkomafont{pagenumber}{\small\sffamily} 	%Font für Seitennummer
%\KOMAoptions{headsepline = true} %Kopzeile Linie
%
%
%\lehead[]{\rightmark} %Zweiseitig
%\cohead[]{}
%\rohead[]{\leftmark\enskip \raisebox{0.7pt}{|}\enskip \thechapter} %Darstellung Kapitel
%
%\lefoot[\pagemark]{\pagemark}
%\lofoot[]{}
%\cofoot[]{}
%\rofoot[\pagemark]{\pagemark} %Seitenzahl


% ============= Literatur =============

\usepackage[									
natbib=true,									% Naturwissenschaftlich
backend=biber,									% Biber statt BibTex verwenden. In TexStudio einstellen!
style=numeric,									% Stil im Verzeichnis alt. authoryear
citestyle=numeric,	                            % Stil im Fließtext
giveninits=true,										
doi=false,										% doi nicht mit angeben
isbn=false,										% isbn nicht mit angeben
sorting=nyt,									% Sortierung des Verzeichnis nach Autor, Titel, Jahr
% sorting=none,									% Keine Sortierung des Verzeichnisses
]{biblatex}	

% \DeclareNameAlias{default}{family-given}
\addbibresource{Literatur.bib} %Bibliographiedateien laden

\urlstyle{same}	% url Schriftart anpassen auf die des Dokuments anpassen
\renewcommand*{\bibfont}{\sffamily} %Schriftart des Literaturverzeichnises
\usepackage{xpatch}% author in bold font in bibliography
\xpretobibmacro{author}{\mkbibbold\bgroup}{}{}
\xapptobibmacro{author}{\egroup}{}{}
\xpretobibmacro{bbx:shortauthor}{\mkbibbold\bgroup}{}{}
\xapptobibmacro{bbx:shortauthor}{\egroup}{}{}
\xpretobibmacro{bbx:editor}{\mkbibbold\bgroup}{}{}
\xapptobibmacro{bbx:editor}{\egroup}{}{}


% ============= Besondere Abkürzungen ============= 

\usepackage[figurename=Abb., % Beschriftung von Abbildung zu Abb.
tablename=Tab., % Beschriftung von Tabelle zu Tab.
labelfont=bf]{caption} %Label in Bold
\KOMAoptions{numbers = noenddot} % der Punkt hinter den Kapiteln etc wird entfernt: 1.1. wird zu 1.1



% ============= Besondere Trennungen ============= 

\hyphenation{De-zi-mal-tren-nung}


% ============= Testing ============= 

\usepackage{blindtext}	% Beispiel Texte einfügen per \blindtext	


%%%%%%%%%%%%%%%%%%%%%%%%%%%%%%%%%%%%%%%%%%%%
% ============= Dokumentbeginn =============
%%%%%%%%%%%%%%%%%%%%%%%%%%%%%%%%%%%%%%%%%%%%

\begin{document}

\pagestyle{empty} %Seiten ohne Kopf- und Fußzeile sowie Seitenzahl
\begin{titlepage}
	\newgeometry{left=2.5cm,right=2cm,top=2.5cm,bottom=1.5cm}
	%------------------------------------------------------------
	\begin{minipage}[t]{60mm}
		\begin{flushleft}
			\hspace{2.0pt}\includegraphics[width=25mm]{Images/Titelseite/ILEK-logo}
		\end{flushleft}
	\end{minipage}
	\begin{minipage}[t]{70mm}
		\begin{flushleft}
			{\fontsize{14}{20}\sffamily{\arbeit}}
			%{\large\textsf\arbeit}
		\end{flushleft}	
	\end{minipage}
	\begin{minipage}[t]{33mm}
		\begin{flushright}
			{\fontsize{14}{20}\sffamily{\Nummer}~|~\textsf{\EndeJahr}}
		\end{flushright}
	\end{minipage}
	%------------------------------------------------------------
	\begin{center} 
		\vspace{-16.0pt}\nointerlineskip\rule{\textwidth}{0.4pt}\\ 
		\vspace{2.0pt}\nointerlineskip
	\end{center}
	%------------------------------------------------------------
	
	\vspace{10mm}
	\hspace{60mm}
	\begin{minipage}[c]{105mm}
		\begin{minipage}[t][4cm][c]{\textwidth}
			\LARGE{\textsf{\titel}}
		\end{minipage}
		\hspace{0mm} 
		
		\vspace{5mm}
		\hspace{-3.0mm} 
		\begin{tabular}{p{2.5cm}l}
			\fontsize{12}{20}\textsf{Bearbeiter:}&\textsf{\student} \\
		\end{tabular}
		
		\vspace{5mm}
		\hspace{-3.0mm} 
		\begin{tabular}{p{2.5cm}l}
			\fontsize{12}{20}\textsf{Betreuer:}&\textsf{\bereuer} \\ 
		\end{tabular}
		
		\vspace{5mm}
		\hspace{-3.0mm} 
		\begin{tabular}{p{2.5cm}l}
			\fontsize{12}{20}\textsf{Prüfer:}&\textsf{\pruefer}	\\
		\end{tabular}
		
		\vspace{11mm}
		\fontsize{12}{20}\textsf{\EndeMonatTxt~\EndeJahr}
		
		
		\begin{minipage}[t][10.5cm][c]{\textwidth}
			\includegraphics[width=1.0\textwidth]{\titelbild}
		\end{minipage}
		
		
		\begin{minipage}[c]{12mm}
			\vspace{-1mm}
			\includegraphics[width=10mm]{Images/Titelseite/unistuttgart_logo_de}
		\end{minipage}
		\begin{minipage}[c]{60mm}
			\fontsize{12}{20}\textsf{Universität Stuttgart}
		\end{minipage}
		
		\fontsize{12}{20}
		\vspace{3mm}
		\textsf{\institut} \\
		\textsf{\profEins} \\
		\textsf{\profZwei} \\
		\textsf{\profDrei} 
		
	\end{minipage}
	\cleardoublepage
\end{titlepage}
%------------------------------------------------------------

\restoregeometry

\pagestyle{plain.scrheadings} % Leere Kopf- und Fußzeilen
\pagenumbering{Roman} % Römische Seitenzahlen verwenden
\include{02_aufgabenstellung} %->Optional
\include{03_erklaerung}
\chapter*{Vorwort}
\addcontentsline{toc}{chapter}{Vorwort}
\label{danksagungen}

\begin{description}
	\item[-] Diese Vorlage dient als grober Leitfaden zu Erstellung der Abschlussarbeit. Die Formatierung ist somit nicht zwingend umzusetzen.
	\item[-] Die Formatierung des Deckblattes sollte, soweit möglich, unverändert bleiben. 
	\item[-] Von der Gliederung der Arbeit kann abgewichen werden, solang dieses sinnig begründbar ist.
\end{description}
\vspace{10mm}
Um mit \LaTeX{} zu Arbeiten, kann z.B. die Kombination folgende Programme verwendet werden.
\begin{flalign*}
&\text{1) MiKTeX:} &&\text{\url{https://miktex.org/download}}&\\
&\text{2) TeXstudio:} &&\text{\url{https://www.texstudio.org/}}
\end{flalign*}
Alternativ besteht auch die Möglichkeit Online-Dienste zu benutzen, welche mögliche Schwierigkeiten bei der Einrichtung umgehen.

\subsection*{Empfohlene Einstellungen dieser Vorlage}
Für eine problemlose Kompilierung des \LaTeX-Dokumentes ist es notwendig, einige Einstellungen in den Editor zu übernehmen.
\begin{description}
	\item[-] Als Standard Bibliographieprogramm sollte Biber ausgewählt werden
	\item[-] Als Standardcompiler ist LuaLaTeX oder PdfLaTeX zu empfehlen
\end{description} %->Optional
\chapter*{Zusammenfassung}\addcontentsline{toc}{chapter}{Zusammenfassung}
\blindtext

\blindtext


\chapter*{Abstract}
\blindtext

\blindtext
\tableofcontents %Inhaltsverzeichnis
\cleardoubleoddpage %Beginn auf einer neuen Seite. Bei Doppelseiten rechts
\pagenumbering{arabic}	% Arabische Seitenzahlen verwenden
\chapter*{Bezeichnungen und Symbole}
\addcontentsline{toc}{chapter}{Bezeichnungen und Symbole}
\begin{longtable}{
		@{}
		>{\centering}p{0.15\linewidth}
		@{}
		>{\hspace*{0pt}}p{0.845\linewidth}
		@{}
	}
	\centering
	\small
	\tabularnewline
	\multicolumn{2}{@{}l}{\textsf{\textbf{Akronyme}}} \\
	APDL & Ansys Parametric Design Language \\
	DMS & Dehnungsmessstreifen \\
	FEM & Finite-Elemente-Methode \\
	
	\tabularnewline
	\multicolumn{2}{@{}l}{\textsf{\textbf{Lateinische Buchstaben}}} \\
	\textit{a} & Erster Eintrag \\
	\textit{b} & Zweiter Eintrag \\
	
	\tabularnewline
	\multicolumn{2}{@{}l}{\textsf{\textbf{Griechische Buchstaben}}} \\
	$\alpha$ & Kontinuierlicher Temperaturabminderungsfaktor \\
	$\varepsilon$ & Dehnung \\
	$\varepsilon_\mathrm{b}$ & Rechnerische Dehnung im Vierpunktbiegeversuch \\
	
	\tabularnewline
	\multicolumn{2}{@{}l}{\textsf{\textbf{Indizes}}} \\
	aktiv & Wert im aktiven Zustand \\
	min & Minimalwert \\
	
\end{longtable}
\cleardoubleoddpage %Beginn auf einer neuen Seite. Bei Doppelseiten rechts
\pagestyle{scrheadings}	% pagestyle für gesamtes Dokument aktivieren (Kopf- und Fußzeilen)


% ============= Kapitel =============
% durch Eigene Kapitel erstezen

\chapter{Einleitung}

\label{sec:einleitung}

\blindtext
\cite{test}

\blindtext
\blindtext
\cite{kroll_computational_2016}

\blindtext
\cite{yang_application_2003}
 % Zitation
\include{11_bsp} % Programmcode
\chapter{Einfügen von Tabellen}
\label{chap:tabellen}

\section{Beispieltabelle}

\begin{table}[h]
	\caption{Beispieltabelle}
	\label{tab:Parameter}
	\centering
	\begin{tabularx}{\textwidth}{p{0.32\textwidth}p{0.32\textwidth}p{0.32\textwidth}}
																					\toprule
		Eins                       	     & Zwei    			& Drei 				\\ 	\midrule
		Vier                       	     & Fünf    			& Sechs				\\
		Sieben                           & Acht         	& Neun				\\ 	\bottomrule	
	\end{tabularx}
\end{table}

\begin{longtabu} to \textwidth {@{} X X[c] X[r] @{}}
	\caption{Tabelle auf Textbreite mit drei gleich großen Spalten}\label{tab:bsp1} \\
	\toprule
	Spalte 1 linksbündig & Spalte 2 zentriert & Spalte 3 rechtsbündig \\
	\midrule
1 , 2                           & 3 , 2                           & 1 , 3 \\
2 , 4                           & 6 , 4                           & 2 , 6 \\
3 , 6                           & 9 , 6                           & 3 , 9 \\
	\bottomrule
\end{longtabu}

\begin{longtabu} to 140mm {@{} X X[c] X[r] @{}}
	\caption{Tabelle auf Textbreite mit drei gleich großen Spalten}\label{tab:bsp2} \\
	\toprule
	Spalte 1 linksbündig & Spalte 2 zentriert & Spalte 3 rechtsbündig \\
	\midrule
1 , 2                           & 3 , 2                           & 1 , 3 \\
2 , 4                           & 6 , 4                           & 2 , 6 \\
3 , 6                           & 9 , 6                           & 3 , 9 \\
	\bottomrule
\end{longtabu}
%

\begin{longtabu} to \textwidth {@{} X[1,l] X[1,c] X[1,r] @{}}
	%----- Kopfzeile erste Tabelle ----- %
	\caption{Tabelle über mehrere Seiten}\label{tab:mehrere Seiten} \\
	\toprule
	Spalte 1 linksbündig & Spalte 2 zentriert & Spalte 3 rechtsbündig \\
	\midrule
	\endfirsthead
	%----- Kopfzeile zweite Tabelle ----- %	
	\caption*{\textbf{Fortsetzung:} \cref{tab:mehrere Seiten}} \\
	\toprule
	Spalte 1 linksbündig & Spalte 2 zentriert & Spalte 3 rechtsbündig \\
	\midrule
	\endhead
	%----- Tabellenende erste Tabelle ----- %	
	\bottomrule
	\endfoot
	%----- Tabellenende zweite Tabelle ----- %
	\bottomrule    
	\endlastfoot
	%----- Inhalt der Tabelle Tabelle ----- %	
1 , 2                           & 3 , 2                           & 1 , 3 \\
2 , 4                           & 6 , 4                           & 2 , 6 \\
3 , 6                           & 9 , 6                           & 3 , 9 \\
4 , 8                           & 12 , 8                           & 4 , 12 \\
5 , 10                           & 15 , 10                           & 5 , 15 \\
6 , 12                           & 18 , 12                           & 6 , 18 \\
7 , 14                           & 21 , 14                           & 7 , 21 \\
8 , 16                           & 24 , 16                           & 8 , 24 \\
9 , 18                           & 27 , 18                           & 9 , 27 \\
10 , 20                           & 30 , 20                           & 10 , 30 \\
11 , 22                           & 33 , 22                           & 11 , 33 \\
12 , 24                           & 36 , 24                           & 12 , 36 \\
13 , 26                           & 39 , 26                           & 13 , 39 \\
14 , 28                           & 42 , 28                           & 14 , 42 \\
15 , 30                           & 45 , 30                           & 15 , 45 \\
16 , 32                           & 48 , 32                           & 16 , 48 \\
17 , 34                           & 51 , 34                           & 17 , 51 \\
18 , 36                           & 54 , 36                           & 18 , 54 \\
19 , 38                           & 57 , 38                           & 19 , 57 \\
\end{longtabu} % Tabellen
\include{13_bsp} % Formeln
\chapter{tikz - Grafiken}
\label{sec:tikz}

\section{Beispielkapitel tikz - Grafiken}
\label{sec:beispiel}

\blindtext

\begin{figure}[!htbp]
\centering
\begin{align*}
f(x)=\tanh(x)
\end{align*}
\begin{tikzpicture}
\begin{axis}
[
width=200pt,
height=200pt,
axis x line=middle, xmin=-4, xmax=4, xtick={-3,...,3}, xlabel=$x$,
axis y line=middle, ymin=-1.5, ymax=1.5, ytick={-1,...,1}, ylabel=$f(x)$,
scale only axis=true,
samples=101
]
\addplot[blue,mark=none, very thick]{tanh(x)};
%		\legend{$\tanh(x)$}
\end{axis}
\end{tikzpicture}
\caption[Tangens hyperbolicus Aktivierungsfunktion]{Tangens hyperbolicus}
\label{fig:AktivTan}
\end{figure}

\blindtext


\begin{figure}[!htbp]
	\centering
	\begin{tikzpicture}[]
	\tikzstyle{netnode}=[circle, inner sep=0pt, text width=22pt, align=center, line width=1.0pt]
	\tikzstyle{inputnode}=[netnode, fill=lightgray,draw=black]
	\tikzstyle{hiddennode}=[netnode, fill=lightgray,draw=black]
	\tikzstyle{outputnode}=[netnode, fill=lightgray,draw=black]
	\tikzstyle{signal}=[arrows={-latex},draw=black,line width=1pt,rounded corners=4pt]
	
	\def\nodedist{35pt}
	\def\layerdist{80pt}
	\def\pindist{20pt}
	
	\tikzstyle{every pin edge}=[signal]
	\tikzstyle{annot} = [text width=6em, text centered]
	
	\foreach \y in {1,...,3}
	\node[inputnode, pin={[pin edge={latex-}, pin distance=\pindist]left:Eingabe \y}] 
	(I\y) at (0,-\y*\nodedist) {$i_\y$};  
	
	\foreach \y in {1,...,4}
	\node[hiddennode] 
	(H1\y) at ($(\layerdist,-\y*\nodedist) +(0, 0.5*\nodedist)$) {$h_\y^1$};
	
	\foreach \y in {1,...,4}
	\node[hiddennode] 
	(H2\y) at ($(2*\layerdist,-\y*\nodedist) +(0, 0.5*\nodedist)$) {$h_\y^2$};
	
	\foreach \y in {1,...,2}
	\node[outputnode, pin={[pin edge={-latex}, pin distance=\pindist]right:Ausgabe \y}]
	(O\y) at ($(H21) + (\layerdist, -\y*\nodedist)$) {$o_\y$};
	
	\foreach \dest in {1,...,4}
	\foreach \source in {1,...,3}
	\draw[signal] (I\source) -- (H1\dest);
	
	\foreach \dest in {1,...,4}
	\foreach \source in {1,...,4}
	\draw[signal] (H1\source) -- (H2\dest);
	
	\foreach \dest in {1,...,2}
	\foreach \source in {1,...,4}
	\draw[signal] (H2\source) edge (O\dest);
	
	\node[annot, above=4pt of H11] (hl) {verborgene Schicht 1};
	\node[annot, above=4pt of H21] (hl) {verborgene Schicht 2};
	\node[annot] at (I1 |- hl) {Eingabe\-schicht};
	\node[annot] at (O1 |- hl) {Ausgabe\-schicht};
	\end{tikzpicture}
	\caption{Schematischer Aufbau eines künstlichen neuronalen Netzes \cite[Abb. nach][]{Frochte.2019}}
	\label{fig:DNN}
\end{figure}


\section{Beispielkapitel Standard Grafik}
\label{sec:graf}

\blindtext

\begin{figure}[h]
  \centering  
  	\includegraphics[scale=0.5]{Images/ILEK-logo.jpg}
  \caption{ILEK Logo}
  \label{fig:starwars}
\end{figure}

\blindtext

 % Grafiken


% ============= Anhang =============

%\appendix
%\begin{appendices}
%	\include{15_bsp}
%\end{appendices}


% ============= Verzeichnise =============

\printbibliography[title=Literaturverzeichnis] %Literaturverzeichnis
\addcontentsline{toc}{chapter}{Literaturverzeichnis}

\listoffigures %Verzeichnis aller Bilder
\addcontentsline{toc}{chapter}{Abbildungsverzeichnis}

\listoftables %Verzeichnis aller Tabellen
\addcontentsline{toc}{chapter}{Tabellenverzeichnis}


% ============= Dokumentende =============

\end{document}
