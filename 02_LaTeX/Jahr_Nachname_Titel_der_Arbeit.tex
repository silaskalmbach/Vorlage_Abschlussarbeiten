%------------------------------------------------------------------------------
%Description:  LaTeX Thesis Template main file
%Author:       silas.kalmbach@ilek.uni-stuttgart.de
%Created:      2021-03-10
%------------------------------------------------------------------------------


% ============= Einstellungen zur Arbeit =============

\documentclass[a4paper,12pt, openright, 
%twoside
]{scrreprt}

\def\tikzextern{0} % 0 = Nein; 1 = Ja (TikZ Grafiken extern Speichern)
\def\mintedinstall{0} % 0 = Nein; 1 = Ja; Um Minted zu nutzen muss Python und Pygments installiert sein. Alternative: listings

\include{00_preamble.tex} % Import relvenanter Einstellungen und Abhängigkeiten


% ============= Angaben zur Arbeit =============

\newcommand{\titel}{Xxxxx Xxxxx Xxxxx Xxxxx Xxxxx Xxxxx Xxxxx Xxxxx Xxxxx Xxxxx Xxxxx} % Hier den Titel der Arbeit angeben 
\newcommand{\arbeit}{MASTERARBEIT} % Hier die Art der Abschlussarbeit eingeben in Großbuchstaben 
\newcommand{\Nummer}{XX} % Nummerrierung der Arbeit (mit Berteurer zu klären)
\newcommand{\EndeMonatTxt}{Xxxxx} % Angabe des Monates der Abgabe als Wort
\newcommand{\EndeJahr}{20XX} % Angabe des Jahres der Abgabe
\newcommand{\bereuer}{Xxxxx Xxxxx} % Hier den Betreuer der Arbeit angeben
\newcommand{\pruefer}{Xxxxx Xxxxx} % Hier den Prüfer der Arbeit angeben 
\newcommand{\student}{Xxxxx Xxxxx} %Hier den Studierendenname der Arbeit angeben 
\newcommand{\profEins}{Prof. Dr.-Ing. M.Arch. Lucio Blandini} % Hier den Professor1 des Instituts angeben 
\newcommand{\profZwei}{Prof. Dr.-Ing. Dr.-Ing. E.h. Dr. h.c. Werner Sobek} % Hier den Professor2 des Instituts angeben 
\newcommand{\profDrei}{Prof. Dr.-Ing. Balthasar Nov\'{a}k} % Hier den Professor3 des Instituts angeben 
\newcommand{\institut}{Institut f{\"u}r Leichtbau Entwerfen und Konstruieren} % Hier das Institut angeben
\newcommand{\titelbild}{Images/Titelseite/Titelbild} % Hier den Pfad des Titelbildes eingeben


% ============= Besondere Trennungen ============= 

\hyphenation{De-zi-mal-tren-nung}



%%%%%%%%%%%%%%%%%%%%%%%%%%%%%%%%%%%%%%%%%%%%
% ============= Dokumentbeginn =============
%%%%%%%%%%%%%%%%%%%%%%%%%%%%%%%%%%%%%%%%%%%%

\begin{document}

\pagestyle{empty} %Seiten ohne Kopf- und Fußzeile sowie Seitenzahl
\include{01_titel}
\restoregeometry

\pagestyle{plain.scrheadings} % Leere Kopf- und Fußzeilen
\pagenumbering{Roman} % Römische Seitenzahlen verwenden
\include{02_aufgabenstellung} %->Optional
\include{03_erklaerung}
\include{04_vorwort} %->Optional
\include{05_zusammenfassung}
\tableofcontents %Inhaltsverzeichnis
\cleardoubleoddpage %Beginn auf einer neuen Seite. Bei Doppelseiten rechts
\pagenumbering{arabic}	% Arabische Seitenzahlen verwenden
\include{06_symbole}
\cleardoubleoddpage %Beginn auf einer neuen Seite. Bei Doppelseiten rechts
\pagestyle{scrheadings}	% pagestyle für gesamtes Dokument aktivieren (Kopf- und Fußzeilen)


% ============= Kapitel =============
% durch Eigene Kapitel erstezen

\include{10_bsp} % Zitation
\include{11_bsp} % Programmcode
\include{12_bsp} % Tabellen
\include{13_bsp} % Formeln
\include{14_bsp} % Grafiken


% ============= Anhang =============

%\appendix
%\begin{appendices}
%	\include{15_bsp}
%\end{appendices}


% ============= Verzeichnise =============

\printbibliography[title=Literaturverzeichnis] %Literaturverzeichnis
\addcontentsline{toc}{chapter}{Literaturverzeichnis}

\listoffigures %Verzeichnis aller Bilder
\addcontentsline{toc}{chapter}{Abbildungsverzeichnis}

\listoftables %Verzeichnis aller Tabellen
\addcontentsline{toc}{chapter}{Tabellenverzeichnis}


% ============= Dokumentende =============

\end{document}
