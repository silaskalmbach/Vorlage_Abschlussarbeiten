%Bei Fragen und Anregungen zu dieser Vorlage können Sie sich gerene bei Silas Kalmbach (silas.kalmbach@ilek.uni-stuttgart.de) melden.

% ============= Einseitig =============
\documentclass[a4paper,12pt]{scrreprt}
\usepackage[left= 2.5cm, right = 2cm, top=2.5cm]{geometry}

% ============= Doppelseitig =============
%\documentclass[a4paper,12pt, openright, twoside]{scrreprt}
%\usepackage[left= 2.5cm, right = 2cm, top=2.5cm]{geometry}
%% -> Anpassungen an die Kopf- und Fußzeilen beachten!


% ============= Dokumentinformationen =============

\usepackage[ 
hidelinks
]{hyperref} 
\hypersetup{ 
	pdfauthor={\student}, 
	pdftitle={\titel},
	pdfsubject={\arbeit}, 
	pdfkeywords={}
}


% ============= Farbschema =============

\usepackage{color}      %Verwendung von Farben
\usepackage{xcolor}		%Definition eigener Farben

% Uni Stuttgart Colors
\definecolor{uniS_lightblue}{RGB}{0,190,255}
\definecolor{uniS_darkblue}{RGB}{0,65,145}
\definecolor{uniS_darkgrey}{RGB}{62, 68, 76}
\definecolor{uniS_greyblue}{RGB}{125,198,234}
\definecolor{uniS_grey}{RGB}{159,153,152}

% Uni Stuttgart monochrom greyscale
\definecolor{uniS_black}{RGB}{0,0,0}
\definecolor{uniS_darkgrey}{RGB}{62,68,76}
\definecolor{uniS_grey}{RGB}{159,153,152}
\definecolor{uniS_lightgrey}{RGB}{185,186,187}


% ============= Standart Packages und Einstellungen =============

%\usepackage[utf8]{inputenc} %Schriftkodierung PDFLaTeX
\usepackage[utf8]{luainputenc} %Schriftkodierung LuaLaTeX
\usepackage[english, ngerman]{babel} %Sprachen
\usepackage{graphicx} %Einbinden von Bildern
\usepackage{amsfonts} % zusätzliche Schriftzeichen der American Mathematical Society
\usepackage{amsmath,amssymb} % zusätzliche Schriftzeichen der American Mathematical Society
\usepackage{typearea} %Aufteilung von Rändern und Textbereich
%\usepackage{minted} %Highlighting von Code, Benötigt Python ,cmd Eingabe->pdflatex -synctex=1 -interaction=nonstopmode --shell-escape %.tex
\usepackage{tikz} %Grafiken Zeichnen
\usetikzlibrary{shapes,arrows,positioning,calc,chains,scopes}
\usepackage{pgfplots}  %Kurven, Graphen
%\usepgfplotslibrary{dateplot, fillbetween, groupplots}
\usepackage{pdfpages} %Einfügen von PDFs
\usepackage{appendix} %Anhang
\usepackage{csquotes}
\usepackage[german]{cleveref} % Mehr Möglichkeiten zum referenzieren
\usepackage{gensymb} %Symbole: \degree \celsius \perthousand \ohm \micro
\tolerance = 400  % möglicher Abstand zwischen den Wörtern erhöhen
\usepackage[section]{placeins} %\FloatBarrier Verhindert das Abbildungen in nachfolgenden Kapitel rutschen
\makeatletter\AtBeginDocument{\expandafter\renewcommand\expandafter\subsection\expandafter{\expandafter\@fb@secFB\subsection}}\makeatother %FloatBarrier für Subsection
\usepackage{subcaption} % Bilder nebeneinander
\captionsetup[subfigure]{list=true, labelfont=bf, position=top}

% ============= Tabelleneinstellungen =============

\usepackage{multirow}						% ermoeglicht mehrere Zeilen in einer Tabellenzeile
\usepackage{multicol} 						% mehre Spalten auf eine Seite
\usepackage{longtable, tabu}				% alternatives Tabellenpackage inkl. longtabu und Farben
\usepackage{float}							% um [H] bei float Objekten verwenden zu können
\usepackage{booktabs}						% um \toprule, \midrule und \bottomrule verwenden zu können
\usepackage{tabularx}
\usepackage{csvsimple} %Excel Improtieren als CSV 
\usepackage{siunitx} %Wissenschaftliche Angabe der Einheiten \num{number}
%für Excel to Latex Add-on
\usepackage{colortbl} %Farbige Tabellen
\usepackage{rotating} %Rotieren von Tabellen
\usepackage{bigstrut} %Für Tabellen mit Vertikalen Linien


% ============= Standart Einstellungen der Pfade =============

\graphicspath{{./Images/}{./Images/Titelseite/}} %Bilderordner


% ============= Schriftarten =============
%PDF-LaTeX
%\usepackage[T1]{fontenc} %Richtige Trennung von Umlauten
%\usepackage[scale=0.87]{tgheros}
%\usepackage{fourier}

%Lua-LaTeX
\usepackage{fontspec}

\setmainfont{UniversforUniS}[
Extension      = .ttf,
Path           = fonts/,
UprightFont    = *55Rm-Regular,
ItalicFont     = *45LtObl-Rg,
BoldFont       = *65Bd-Regular,
BoldItalicFont = *45LtObl-Rg,
]

\setsansfont{UniversforUniS}[
Extension      = .ttf,
Path           = fonts/,
UprightFont    = *55Rm-Regular,
ItalicFont     = *45LtObl-Rg,
BoldFont       = *65Bd-Regular,
BoldItalicFont = *45LtObl-Rg,
]

%\setmonofont{CMU Typewriter Text}

\addtokomafont{disposition}{\rmfamily} %Überschriften selbe Schriftart wie Text

\usepackage{setspace} %Zeilenabstand
\setstretch{1.05} %Zeilenabstand

\setlength{\parindent}{0pt} %Wenn Absatzabstand, dann Einzug unnötig
\setlength{\parskip}{7pt} %Absatzabstand

\setkomafont{captionlabel}{\sffamily\bfseries} %Schrift der Labels
\setkomafont{caption}{\sffamily} %Schrift der Labelbezeichnung
\usepackage[eulergreek]{sansmath} %Schrift in Diagrammen
\AtBeginEnvironment{longtabu}{\sffamily} %Schriftart Tabelle %\usepackage{etoolbox}
\AtBeginEnvironment{figure}{\sffamily} %Schriftart Figure
\AtBeginEnvironment{tabu}{\sffamily} %Schriftart Tabelle
\AtBeginEnvironment{tikzpicture}{\footnotesize\sffamily} %Schriftart Tabelle \small, \footnotesize


% ============= Verzeichnisse =============

\RedeclareSectionCommands[
tocentryformat=\sffamily,
tocpagenumberformat=\sffamily
]{part,section,subsection,subsubsection,paragraph,subparagraph}

\DeclareTOCStyleEntry[pagenumberformat=\sffamily, entryformat=\sffamily]{tocline}{table}
\DeclareTOCStyleEntry[pagenumberformat=\sffamily, entryformat=\sffamily]{tocline}{figure}


% ============= Kopf- und Fußzeile >>Einseitig<< =============

\usepackage{scrlayer-scrpage}
\automark{chapter}	%Automatische Kapitelangabe in Kopfzeile
\renewcommand*{\chaptermarkformat}{} %Kapitel ohne Kapitelnummer
\clearpairofpagestyles %aktuelle Einetsllungen aus Kopf- und Fußzeile löschen
\setkomafont{pageheadfoot}{\small\sffamily}		%Font für Kopfzeile
\setkomafont{pagefoot}{\small\sffamily}  		%Font für Fußzeile
\setkomafont{pagenumber}{\small\sffamily} 	%Font für Seitennummer
\KOMAoptions{headsepline = true} %Kopzeile Linie

\lohead[]{}
\cohead[]{}
\rohead[]{\headmark\enskip \raisebox{0.7pt}{|}\enskip \thechapter} %Darstellung Kapitel

\lofoot[]{}
\cofoot[]{}
\rofoot[\pagemark]{\pagemark} %Seitenzahl


% ============= Kopf- und Fußzeile >>Doppelseitig<< =============

%\usepackage[automark, headsepline, autooneside=false]{scrlayer-scrpage}
%\automark[section]{chapter}	%Automatische Kapitelangabe in Kopfzeile
%\renewcommand*{\chaptermarkformat}{} %Kapitel ohne Kapitelnummer
%\renewcommand*{\sectionmarkformat}{} %Kapitel ohne Kapitelnummer
%\clearpairofpagestyles %aktuelle Einetsllungen aus Kopf- und Fußzeile löschen
%\setkomafont{pageheadfoot}{\small\sffamily}		%Font für Kopfzeile
%\setkomafont{pagefoot}{\small\sffamily}  		%Font für Fußzeile
%\setkomafont{pagenumber}{\small\sffamily} 	%Font für Seitennummer
%\KOMAoptions{headsepline = true} %Kopzeile Linie
%
%
%\lehead[]{\rightmark} %Zweiseitig
%\cohead[]{}
%\rohead[]{\leftmark\enskip \raisebox{0.7pt}{|}\enskip \thechapter} %Darstellung Kapitel
%
%\lefoot[\pagemark]{\pagemark}
%\lofoot[]{}
%\cofoot[]{}
%\rofoot[\pagemark]{\pagemark} %Seitenzahl


% ============= Literatur =============

\usepackage[									
natbib=true,									% Naturwissenschaftlich
backend=biber,									% Biber statt BibTex verwenden. In TexStudio einstellen!
style=numeric,									% Stil im Verzeichnis alt. authoryear
citestyle=numeric,	                            % Stil im Fließtext
giveninits=true,										
doi=false,										% doi nicht mit angeben
isbn=false,										% isbn nicht mit angeben
sorting=nyt,									% Sortierung des Verzeichnis nach Autor, Titel, Jahr
% sorting=none,									% Keine Sortierung des Verzeichnisses
]{biblatex}	

% \DeclareNameAlias{default}{family-given}
\addbibresource{Literatur.bib} %Bibliographiedateien laden

\urlstyle{same}	% url Schriftart anpassen auf die des Dokuments anpassen
\renewcommand*{\bibfont}{\sffamily} %Schriftart des Literaturverzeichnises
\usepackage{xpatch}% author in bold font in bibliography
\xpretobibmacro{author}{\mkbibbold\bgroup}{}{}
\xapptobibmacro{author}{\egroup}{}{}
\xpretobibmacro{bbx:shortauthor}{\mkbibbold\bgroup}{}{}
\xapptobibmacro{bbx:shortauthor}{\egroup}{}{}
\xpretobibmacro{bbx:editor}{\mkbibbold\bgroup}{}{}
\xapptobibmacro{bbx:editor}{\egroup}{}{}


% ============= Besondere Abkürzungen ============= 

\usepackage[figurename=Abb., % Beschriftung von Abbildung zu Abb.
tablename=Tab., % Beschriftung von Tabelle zu Tab.
labelfont=bf]{caption} %Label in Bold
\KOMAoptions{numbers = noenddot} % der Punkt hinter den Kapiteln etc wird entfernt: 1.1. wird zu 1.1
